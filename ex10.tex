\documentclass{llncs}
\usepackage{amssymb}
\usepackage{amsmath}
\usepackage{graphicx}
\usepackage{gensymb}
\usepackage{epstopdf}
\usepackage{hyperref}
\usepackage{circus}

\begin{document}
\parindent 0pt
\parskip 1.2ex plus 4pt

\section{The birthday book}\label{bb}

The best way to see how these ideas work out is to look at a
small example. For a first example, it is important to
choose something simple, and I have chosen a system so
simple that it is usually implemented with a notebook and
pencil rather than a computer. It is a system which records
people's birthdays, and is able to issue a reminder when the
day comes round.

In our account of the system, we shall need to deal with people's
names and with dates. For present purposes, it will not matter what
form these names and dates take, so we introduce the set of all names
and the set of all dates as {\em basic types\/} of the specification:
\begin{zed} [NAME, DATE]\end{zed}
This allows us to name the sets without saying what kind of objects
they contain.
The first aspect of the system to describe is its {\em state space\/},
and we do this with a schema:
\begin{schema}{BirthdayBook}
    known: \power NAME \\
    birthday: NAME \pfun DATE
\where
    known=\dom birthday
\end{schema}
Like most schemas, this consists of a part above the central
dividing line, in which some variables are declared, and a
part below the line which gives a relationship between the
values of the variables. In this case we are describing the
state space of a system, and the two variables represent
important {\em observations\/} which we can make of the
state:
\begin{itemize}
\item $known$ is the set of names with birthdays recorded;
\item $birthday$ is a function which, when applied to certain
names, gives the birthdays associated with them.
\end{itemize}
The part of the schema below the line gives a relationship
which is true in every state of the system and is maintained
by every operation on it: in this case, it says that the set
$known$ is the same as the domain of the function $birthday$
-- the set of names to which it can be validly applied.
This relationship is an {\em invariant\/} of the system.

In this example, the invariant allows the value of the variable
$known$ to be derived from the value of $birthday$: $known$ is a {\em
derived\/}\index{derived component}%
	\glossary{[derived component] A component of a schema
	describing the state space of an abstract data type whose
	value can be deduced from the values of the other components.}
component of the state, and it would be possible to
specify the system without mentioning $known$ at all.  However,
giving names to important concepts helps to make specifications more
readable; because we are describing an abstract view of the state space
of the birthday book, we can do this without making a
commitment to represent $known$ explicitly in an implementation.

One possible state of the system has three people in the set $known$,
with their birthdays recorded by the function $birthday$:
\[
	known = \{\,{\rm John, Mike, Susan}\,\} \\
\also
	birthday = \{\,\vtop{\halign{\strut#\hfil&${}\mapsto{}$#\hfil\cr
			John&  25--Mar,\cr
			Mike&  20--Dec,\cr
			Susan& 20--Dec\,\}.\cr}}
\]
The invariant is satisfied, because $birthday$ records a date for
exactly the three names in $known$.

Notice that in this description of the state space of the
system, we have not been forced to place a limit on the
number of birthdays recorded in the birthday book,
nor  to say that the
entries will be stored in a particular order. We have also
avoided making a premature decision about the format of
names and dates. On the other hand, we have concisely
captured the information that each person can have only one
birthday, because the variable $birthday$ is a function, and
that two people can share the same birthday as in our
example.

So much for the state space; we can now start on some {\em
operations\/} on the system. The first of these is to add a
new birthday, and we describe it with a schema:
\begin{schema}{AddBirthday}
     \Delta BirthdayBook \\
     name?: NAME \\
     date?: DATE
\where
     name? \notin known
\also
     birthday' = birthday \cup \{name? \mapsto date?\}
\end{schema}
The declaration $\Delta BirthdayBook$\index{Delta convention} alerts
us to the fact that the schema is describing a {\em state change}: it
introduces four variables $known$, $birthday$, $known'$ and
$birthday'$. The first two are observations of the state before the
change, and the last two are observations of the state after the
change. Each pair of variables is implicitly constrained to satisfy
the invariant, so it must hold both before and after the operation.
Next come the declarations of the two inputs to the operation.
By convention, the names of inputs end in a question mark.

The part of the schema below the line first of all gives a {\em
pre-condition\/}\index{pre-condition} for the success of the
operation: the name to be added must not already be one of those known
to the system. This is reasonable, since each person can only have one
birthday. This specification does not say what happens if the
pre-condition is not satisfied: we shall see later how to extend the
specification to say that an error message is to be produced. If the
pre-condition is satisfied, however, the second line says that the
birthday function is extended to map the new name to the given date.

We expect that the set of names known to the system will be
augmented with the new name:
\[ known' = known \cup \{name?\}. \]
In fact we can {\em prove\/} this from the specification of
$AddBirthday$, using the invariants on the state before and
after the operation:
\begin{argue}
	known' = \dom birthday' &		invariant after \\
\t1    	= \dom (birthday \cup \{name? \mapsto date?\}) &
						spec.\ of $AddBirthday$ \\
\t1	= \dom birthday \cup \dom\,\{name? \mapsto date?\} &
						fact about `$\dom$' \\
\t1	= \dom birthday \cup \{name?\} &	fact about `$\dom$' \\
\t1	= known \cup \{name?\}. & 		invariant before
\end{argue}
Stating and proving properties like this one is a good way
of making sure the specification is accurate; reasoning from
the specification allows us to explore the behaviour
of the system without going to the trouble and expense of
implementing it.
The two facts about `$\dom$' used in this proof are examples
of the laws obeyed by mathematical data types:
\[
	\dom (f \cup g) = (\dom f) \cup (\dom g) \\
\also
	\dom \{a \mapsto b\} = \{a\}.
\]
Chapter~\ref{c:library} contains many laws like these.

Another operation might be to find the
birthday of a person known to the system. Again we describe
the operation with a schema:
\begin{schema}{FindBirthday}
	\Xi BirthdayBook \\
	name?: NAME \\
	date!: DATE
\where
	name? \in known
\also
	date! = birthday(name?)
\end{schema}
This schema illustrates two new notations.  The
declaration $\Xi BirthdayBook$\index{Xi convention} indicates that
this is an operation in which the state does not change: the values $known'$
and $birthday'$ of the observations after the operation are equal to
their values $known$ and $birthday$ beforehand. Including $\Xi
BirthdayBook$ above the line has the same effect as including $\Delta
BirthdayBook$ above the line and the two equations
\[
	known' = known
\also
	birthday' = birthday
\]
below it. The other notation is the use of a name ending in
an exclamation mark for an output: the
$FindBirthday$ operation takes
a name as input and yields
the corresponding birthday as output.
\pagebreak[1]
The pre-condition for success
of the operation is that
$name?$ is one of the names known to the system; if this is
so, the output $date!$ is the value of the birthday function
at argument $name?$.

The most useful operation on the system is the one to find
which people have birthdays on a given date.  The operation has one
input $today?$, and one output, $cards!$, which is a {\em set\/} of names:
there may be zero, one, or more people with birthdays on a
particular day, to whom birthday cards should be sent.
\begin{schema}{Remind}
	\Xi BirthdayBook \\
	today?: DATE \\
	cards!: \power NAME
\where
	cards! = \{\,n: known | birthday(n) = today?\,\}
\end{schema}
Again the $\Xi$ convention is used to indicate that the
state does not change. This time there is no pre-condition.
The output $cards!$ is specified to be equal to the set of
all values $n$ drawn from the set $known$ such that the
value of the birthday function at $n$ is $today?$. In general,
$y$ is a member of the set $\{\,x:S | \ldots x \ldots\,\}$
exactly if $y$ is a member of $S$ and the condition $\ldots
y \ldots$, obtained by replacing $x$ with $y$, is satisfied:
\[ y \in \{\,x: S | \ldots x \ldots \,\}
	\iff y \in S \land (\ldots y \ldots). \]
So, in our case,
\[ m \in \{\,n: known | birthday(n) = today?\,\} \\
\t1 		\iff m \in known \land birthday(m) = today?~. \]
A name $m$ is in the output set $cards!$ exactly if it is
known to the system and the birthday recorded for it is
$today?$.

To finish the specification, we must say what state the system is in
when it is first started. This is the {\em initial state\/} of the
system, and it also is specified by a schema:
\begin{schema}{InitBirthdayBook}
	BirthdayBook
\where
	known = \emptyset
\end{schema}
This schema describes a birthday book in which the set $known$ is
empty: in consequence, the function $birthday$ is empty too.

What have we achieved in this specification? We have described in the
same mathematical framework both the state space of our birthday-book
system and the operations which can be performed on it.  The data
objects which appear in the system were described in terms of
mathematical data types such as sets and functions. The description of
the state space included an invariant relationship between the parts of
the state -- information which would not be part of a program
implementing the system, but which is vital to understanding it.

The effects of the operations are described in terms of the
relationship which must hold between the input and the
output, rather than by giving a recipe to be followed. This
is particularly striking in the case of the $Remind$
operation, where we simply documented the conditions under
which a name should appear in the output. An implementation
would probably have to examine the known names one at a
time, printing the ones with today's date as it found them,
but this complexity has been avoided in the specification. The
implementor is free to use this technique, or any other one,
as he or she chooses.

% Mathematical specifications have the three virtues of being concise,
% precise and unambiguous.  They are {\em concise\/} because mathematical
% notation is capable of expressing complex facts about information
% systems in a short space.  Practical experience shows that a
% mathematical specification of a system is often much shorter than an
% equivalent informal specification.
% Mathematical specifications are {\em precise\/} because they allow
% requirements to be documented accurately. The desired function of a
% system is described in a way that does not unduly constrain either the
% data structures used to represent the information in the system, or the
% algorithms used to compute with it.  Finally, mathematical
% specifications are {\em unambiguous}: differences of interpretation can
% be avoided when specifications are expressed in a standardized language
% with a well-understood meaning.

\section{Strengthening the specification}

A correct implementation of our specification will faithfully record
birthdays and display them, so long as there are no mistakes in the
input. But the specification has a serious flaw: as soon as the user
tries to add a birthday for someone already known to the system, or
tries to find the birthday of someone not known, it says nothing about
what happens next. The action of the system may be perfectly
reasonable: it may simply ignore the incorrect input. On the other
hand, the system may break down: it may start to display rubbish, or
perhaps worst of all, it may appear to operate normally for several
months, until one day it simply forgets the birthday of a rich and
elderly relation.

Does this mean that we should scrap the specification and begin a new
one? That would be a shame, because the specification we have
describes clearly and concisely the behaviour for correct input, and
modifying it to describe the handling of incorrect input could only
make it obscure.  Luckily there is a better solution: we can describe,
separately from the first specification, the errors which might be
detected and the desired responses to them, then use the operations of
the Z {\em schema calculus\/} to combine the two descriptions into a
stronger specification.

We shall add an extra output $result!$ to each operation on the
system.  When an operation is successful, this output will take the
value $ok$, but it may take the other values $already\_known$ and
$not\_known$ when an error is detected. The following {\em free type
definition\/}\index{free type: definition} defines $REPORT$ to be a set
containing exactly these three values:

\begin{zed}
REPORT ::= ok | already\_known | not\_known
\end{zed}
We can define a schema $Success$ which just specifies that the
result should be $ok$, without saying how the state changes:
\begin{schema}{Success}
	result!: REPORT
\where
	result! = ok
\end{schema}
The conjunction operator $\land$ of the schema calculus allows us to
combine this description with our previous description of $AddBirthday$:
\[ AddBirthday \land Success. \]
This describes an operation which, for correct input, both acts as
described by $AddBirthday$ and produces the result $ok$.

For each error that might be detected in the input, we define a schema
which describes the conditions under which the error occurs and
specifies that the appropriate report is produced. Here is a schema
which specifies that the report $already\_known$ should be produced
when the input $name?$ is already a member of $known$:
\begin{schema}{AlreadyKnown}
	\Xi BirthdayBook \\
	name?: NAME \\
	result!: REPORT
\where
	name? \in known \\
	result! = already\_known
\end{schema}
The declaration $\Xi BirthdayBook$ specifies that if the error occurs,
the state of the system should not change.

We can combine this description with the previous one to give a
specification for a robust version of $AddBirthday$:
\[ RAddBirthday \defs (AddBirthday \land Success) \lor AlreadyKnown. \]
This definition introduces a new schema called $RAddBirthday$,
obtained by combining the three schemas on the right-hand side.
The operation $RAddBirthday$ must terminate whatever its input.  If
the input $name?$ is already known, the state of the system does not
change, and the result $already\_known$ is returned; otherwise,
the new birthday is added to the database as described by
$AddBirthday$, and the result $ok$ is returned.

We have specified the various requirements for this operation separately,
and then combined them into a single specification of the whole
behaviour of the operation. This does not mean that each requirement
must be implemented separately, and the implementations combined
somehow.
In fact, an implementation might search for a place to store the new
birthday, and at the same time check that the name is not already known;
the code for normal operation and error handling might be thoroughly
mingled.
This is an example of the abstraction which is possible when we use a
specification language free from the constraints necessary
in a programming language. The operators $\land$ and $\lor$
cannot (in general) be implemented efficiently as ways of combining
programs, but this should not stop us from using them to combine
specifications if that is a convenient thing to do.

The operation $RAddBirthday$ could be specified directly by writing
a single schema which combines the predicate parts of the three
schemas $AddBirthday$, $Success$ and $AlreadyKnown$.
The effect of the schema $\lor$ operator is to make a schema
in which the predicate part is the result of joining the predicate parts of
its two arguments with the logical connective $\lor$. Similarly, the effect
of the schema $\land$ operator is to take the conjunction of the two
predicate parts.
Any common variables of the two schemas are merged: in this example, the
input $name?$, the output $result!$, and the four observations of the
state before and after the operation are shared by the two arguments of
$\lor$.
\begin{schema}{RAddBirthday}
	\Delta BirthdayBook \\
	name?: NAME \\
	date?: DATE \\
	result!: REPORT
\where
name? \notin known\\
birthday' = birthday \cup \{name? \mapsto date?\}\\
result! = ok\\
\end{schema}
In order to write $RAddBirthday$ as a single schema, it has been
necessary to write out explicitly that the state doesn't change when
an error is detected, a fact that was implicitly
part of the declaration $\Xi BirthdayBook$ before.

A robust version of the $FindBirthday$ operation must be able to report
if the input name is not known:
\begin{schema}{NotKnown}
	\Xi BirthdayBook \\
	name?: NAME \\
	result!: REPORT
\where
	name? \notin known \\
	result! = not\_known
\end{schema}
The robust operation either behaves as described by $FindBirthday$ and
reports success, or reports that the name was not known:
\[ RFindBirthday \defs (FindBirthday \land Success) \lor NotKnown. \]
The $Remind$ operation can be called at any time: it never results in
an error, so the robust version need only add the reporting of success:
\[ RRemind \defs Remind \land Success. \]

\looseness=1
The separation of normal operation from error-handling which we
have seen here is the simplest but also the most common kind of
modularization possible with the schema calculus.
More complex modularizations include {\em promotion} or
{\em framing}, where operations
on a single entity -- for example, a file -- are made into
operations on a named entity in a larger system -- for example, a
named file in a directory.
The operations of reading and writing a file might be described by
schemas. Separately, another schema might describe the way a file can
be accessed in a directory under its name. Putting these two parts
together would then result in a specification of operations for
reading and writing named~files.

Other modularizations are possible: for example, the
specification of a system with access restrictions might separate the
description of who may call an operation from the description of what
the operation actually does.
There are also facilities for generic definitions in Z which allow, for
example, the notion of resource management to be specified in general,
then applied to various aspects of a complex system.
\end{document}
