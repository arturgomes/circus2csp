include "platform.csp"
\begin{circus}
LINEAR
 = fi.KNL.NOP.1 ->
   fi.KNL.S0.1  ->
   fi.KNL.S1.0  -> LINEAR

TREE = PC(0)

PC(pc)
 =  pc == 0 & fi.KNL.NOP.1 -> PC(1)
    \extchoice
    pc == 1 & fi.KNL.S0.1  -> PC(2)
    \extchoice
    pc == 2 & fi.KNL.S1.0  -> PC(0)

\end{circus}

{- ========OLD STUFF ===========

-------------
--CODE:
-------------

KERNEL_INIT_CODE(me) =
--fe.me?(x).(jDC).(aDC).(dDC) ->    --run all possible instructions
--fe.me?(x).(jDC).(aDC).(dDC) ->    --run all possible instructions
--fe.me.nopInst.jDC.aDC.dDC ->
--fe.me.setSupModeInst.jDC.a0.d0 ->    --Instruction to set mode to supervisor
--fe.me.nopInst.jDC.aDC.dDC ->

--fe.me.writeInst.jDC.aX.d0 ->     -- a legal write, no memfault

KERNEL_LOOP(me)

KERNEL_LOOP(me) =
fe.me.nopInst.jDC.aDC.dDC ->
KERNEL_LOOP(me)



PARTITION_1(me) =

fe.me.nopInst.jDC.aDC.dDC ->



PARTITION_1_DONE(me)
PARTITION_1_DONE(me) =

fe.me.nopInst.jDC.aDC.dDC ->

PARTITION_1_DONE(me)


PARTITION_2(me) =

PARTITION_2_DONE(me)
PARTITION_2_DONE(me) =
fe.me.nopInst.jDC.aDC.dDC ->
fe.me.writeInst.(jDC)?(x).(d0) ->    --Attempt to write to all possible addresses
PARTITION_2_DONE(me) --loop forever for now



SCHEDULER_HANDLER(me) =
fe.me.mmuBlockInst.jDC.a0.dDC ->   --block a0
fe.me.mmuUnBlockInst.jDC.a1.dDC -> --unblock a1
fe.me.mmuBlockInst.jDC.a2.dDC ->   --block a2
fe.me.setUsrModeInst.jDC.aDC.dDC ->
fe.me.jumpInst.partition1.aDC.dDC ->   --Instruction to jump to Partition1


fe.me.mmuBlockInst.jDC.a0.dDC ->      --block a0
fe.me.mmuBlockInst.jDC.a1.dDC ->      --block a1
fe.me.mmuUnBlockInst.jDC.a2.dDC ->    --unblock a2
fe.me.setUsrModeInst.jDC.aDC.dDC ->
fe.me.jumpInst.partition2.aDC.dDC ->   --Instruction to jump to Partition2
SCHEDULER_HANDLER(me)


MEMFAULT_HANDLER(me) =
--fe.me.setSupModeInst.jDC.a0.d0 ->    --Instruction to set mode to supervisor
fe.me.nopInst.jDC.aDC.dDC ->

fe.me.handlerReturnInst.jDC.aDC.dDC ->    --Instruction to enable IRQ forwarding, return to previous code section + mode by popping stack
MEMFAULT_HANDLER(me)




SUPINST_HANDLER(me) =

fe.me.nopInst.jDC.aDC.dDC ->
fe.me.handlerReturnInst.jDC.aDC.dDC ->    --Instruction to enable IRQ forwarding, return to previous code section + mode by popping stack
SUPINST_HANDLER(me)







SCENARIO_COMPLETED_HANDLER(me) =
SCENARIO_COMPLETED ->
SCENARIO_COMPLETED_HANDLER(me)

CODE0 =  KERNEL_INIT_CODE(kernel)
CODE1 = CODE0 [| {|   |} |] PARTITION_1(partition1)
CODE2 = CODE1 [| {|   |} |] PARTITION_2(partition2)
CODE3 = CODE2 [| {|   |} |] MEMFAULT_HANDLER(mf_handler)
CODE4 = CODE3 [| {|   |} |] SUPINST_HANDLER(supI_handler )
CODE7 = CODE4 [| {|   |} |] SCHEDULER_HANDLER(scheduler )
--CODE8 = CODE7 [| {|   |} |] KERNEL_ERROR_CODE(kernel_error )
--CODE9 = CODE8 [| {|  |} |] SCENARIO_COMPLETED_HANDLER(scenario_completed)

CODE = CODE7


-}
