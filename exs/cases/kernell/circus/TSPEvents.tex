{- =============================================================================

	TSPEvents.csp

 These are the globally visible TSP events
-}


{- Code Identity ---------------------------------------------------------------

  Given configuration data we can identify the code running
  by its range of instruction fetch addresses.
  We abstract this as a type that identifies the relevant entity,
  and by making code fetch observable.

  For now we bake partition authorisation into our entity definition.
-}
\begin{circus}
datatype ENT ::=
  KNL  -- kernel
 | P1   -- partition 1 (authorised to do everything priviledged)
 | P2   -- partition 2 (not authorised to anything priviledged)
subtype PARTITION = P1 | P2
\begin{circus}

{- Code Execution --------------------------------------------------------------

  We model an instruction fetch with an event that records not just the opcode
  and operands, but also the identity of the entity in whose memory space the
  instruction resides.
-}
\begin{circus}

channel fi : ENT.I.OP
\begin{circus}

{- Partition Starting ----------------------------------------------------------

  In any partition code we simply mark the cold-start points (the beginning)
  and the warm-start points (sometime later).
  We also observe the kernel (or authorised partition) starting a partition.

  Note that a switch to/from kernel and a partition can happen without these
  markers - that is a normal context-switch.
-}
\begin{circus}

datatype LBL  -- labels marking interesting code points
  ::= CLD   -- in partition code just before first cold-start instruction
  | WRM   -- in partition code just before first warm-start instruction
\end{circus}

channel lbl : LBL
\begin{circus}

channel st : ENT.LBL  -- start entity, at label
\end{circus}
-- valid partition code has at most one of each.

{- Partition Modes -------------------------------------------------------------
  These are STOPPED, READY, RUNNING, and ACTIVE = READY | RUNNING
-}
\begin{circus}

datatype PMODE = STOPPED | READY | RUNNING
subtype ACTIVE = READY | RUNNING
\end{circus}

{- Hypercalls -------------------------------------------------------------
   We shall model these similarly to instruction fetches
-}
\begin{circus}

datatype HCALL ::= START.PARTITION  -- start partition
\end{circus}
\begin{circus}

channel hc : PARTITION.HCALL  -- always followed by a kernel instruction
\end{circus}

{- We need to define a check process that ensures that a hypercall is always
followed by a kernel instruction -}
\begin{circus}

channel badHCall
\end{circus}
\begin{circus}

HCALLSCAN =     fi?ent?i?op     -> HCALLSCAN
             [] hc?p?call:HCALL -> SEENHCALL
\end{circus}
\begin{circus}
SEENHCALL  =    fi.KNL?i?op     -> HCALLSCAN
             [] hc?p?call:HCALL -> HCALLFAIL
             [] ([] p:PARTITION @ fi.p?i?op -> HCALLFAIL)
\end{circus}
\begin{circus}
HCALLFAIL  = badHCall -> STOP
\end{circus}

HCALLCHECK(CODE) = HCALLSCAN [| {|fi,hc|} |] CODE
-- !!! Is this really part of some Requirement?

{- Kernel Actions --------------------------------------------------------------
  The Kernel can change partition modes,...
-}

channel partmode : PMODE

{- Timings  --------------------------------------------------------------
  We can mark schedule boundaries
-}

datatype SLOTBNDRY = SLOTBEGIN | SLOTEND
channel slot : PARTITION.SLOTBNDRY

{- We should check that we always have slot begin followed by the end of the
same slot  -}


channel badSchedule

-- we assume we start outside any time-slot
SCHEDSCAN   =    slot?p!SLOTBEGIN -> SLOTSCAN(p)
              [] slot?p!SLOTEND   -> SCHEDFAIL
SLOTSCAN(p) =    slot.p.SLOTEND   -> SCHEDSCAN
              [] ([] p:diff(PARTITION,{p}) @ slot.p.SLOTEND -> SCHEDFAIL)
              [] slot.p.SLOTBEGIN -> SCHEDFAIL
SCHEDFAIL   = badSchedule -> STOP

{- We should check that partition code only runs during its own timeslot -}

channel partOutOfTime

-- here we assume that SCHEDSCAN above is satisfied
PTIMESCAN =    fi?KNL?i?op  -> PTIMESCAN
            [] ([] p:PARTITION @ slot.p.SLOTBEGIN -> PSLOT(p) )
            [] ([] p:PARTITION @ fi.p?i?op        -> PTIMEFAIL)
PSLOT(p)  =    fi?KNL?i?op    -> PSLOT(p)
            [] fi.p?i?op      -> PSLOT(p)
            [] slot.p.SLOTEND -> PTIMESCAN
            [] ([] p:diff(PARTITION,{p}) @ fi.p?i?op -> PTIMEFAIL)
PTIMEFAIL = partOutOfTime -> STOP
-- this is definitely one of the other requirements!!!

{- Code Termination ------------------------------------------------------------
   Partition code may end,
   in which case we want to signal that and then busy loop
-}
channel done : ENT

DONE(ent) = done.ent -> NOPLOOP(ent)
NOPLOOP(ent) = fi.ent.NOP.0 -> NOPLOOP(ent)

{- Generic Partition Behaviour -------------------------------------------------

  A partition starts with cold start code, followed by warm start code,
  and then enters a running phase, from which it might decide to terminate.
-}

GP(ent)
 = lbl.CLD -> fi.ent.NOP.0 -> lbl.WRM -> fi.ent.NOP.1 -> GPRUN(ent)

GPRUN(ent)
 = (fi.ent.NOP?op -> GPRUN(ent))
   |~|
   DONE(ent)
