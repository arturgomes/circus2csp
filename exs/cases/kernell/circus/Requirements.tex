{- Requirements Modelling =================================

 Every requirement is modelled as a process over hardware and TSP events,
 with some requirement-specific marker events added in.

 Most requirement models encode the relevant satisfaction property in terms
 of either the presence or absence of deadlock, as appropriate.

-}

% include "Platform.csp"

{- 4.1 Initialisation Management ------------------------- -}

{- PK-79
The partitioning kernel shall start its execution by initialising the system according to the defined configuration.
-}

{- PK-84
After PK initialization phase, the partitioning kernel shall start the execution of the default scheduling plan as defined in configuration.
-}

{- PK-86
The partitioning Kernel configuration shall allow defining the entry start address of each partition, this entry start address being in RAM or in NVR (execute in place partition).
-}

{- PK-101
Before restarting a partition from its entry start address, the partitioning kernel shall initialize its context.
-}

{- 4.2 Partition Mode Management -------------------------

Things we need to observe:
  - what code (kernel, partition1, ...) is currently running
  - what priviledge mode (user, supervisor) the processor is currently in
  - which code segment (initialisation, main loop, ...) of an entity is running,
    and when it starts and finishes.
  - we need to observe when the kernel starts a partition (cold or warm).
-}


{- PK-230
Partitions shall be executed in user mode. (Req 5)
-
We track all ftch events and processor mode changes
-}
\begin{circus}
\circchannel pk230fail

PK230 = PK230sup -- assumes supervisor mode when started

PK230sup
 =  (ftch.KNL?i?op -> PK230sup)
    []
    (ftch.P1?i?op -> pk230fail -> STOP)
    []
    (ftch.P1?i?op -> pk230fail -> STOP)
    []
    (pm.USR -> PK230usr)
    []
    (pm.SUP -> PK230sup)

PK230usr
 =  (ftch?ent?i?op -> PK230usr)
    []
    (pm.USR -> PK230usr)
    []
    (pm.SUP -> PK230sup)

-- we now give two tests: good behaviour and bad behaviour

PK230GOODCODE
 = ftch.KNL.NOP.0 -> ftch.KNL.S0.0 ->
   ftch.P1.NOP.0 -> ftch.P1.NOP.1 ->
   ftch.P1.NOP.0 -> ftch.P1.NOP.1 ->
   ftch.KNL.S1.0 -> ftch.KNL.NOP.1 -> PK230GOODCODE

assert PK230GOODCODE :[divergence free]
assert PK230GOODCODE :[deadlock free[F]]

PK230GOODSYS = PMM [| {| ftch |} |] PK230GOODCODE

assert PK230GOODSYS :[divergence free]
assert PK230GOODSYS :[deadlock free[F]]

PK230GOOD = PK230 [| {| ftch, pm |} |] PK230GOODSYS

assert PK230GOOD :[divergence free]
assert PK230GOOD :[deadlock free[F]]

PK230BADCODE
 = ftch.KNL.NOP.0 -> ftch.KNL.S1.0 ->
   ftch.P1.NOP.0 -> ftch.P1.NOP.1 ->
   ftch.P1.NOP.0 -> ftch.P1.NOP.1 ->
   ftch.KNL.S1.0 -> ftch.KNL.NOP.1 -> PK230BADCODE

assert PK230BADCODE :[divergence free]
assert PK230BADCODE :[deadlock free[F]]

PK230BADSYS = PMM [| {| ftch |} |] PK230BADCODE

assert PK230BADSYS :[divergence free]
assert PK230BADSYS :[deadlock free[F]]

PK230BAD = PK230 [| {| ftch, pm |} |] PK230BADSYS

assert PK230BAD :[divergence free]
assert PK230BAD :[deadlock free[F]]
\end{circus}
{- PK-98
When a partition enters in the ACTIVE mode, the Partitioning Kernel shall start executing the partition at the entry start address as defined in configuration.
-}

{- PK-541
When a partition enters in the ACTIVE mode,  its associated vCPU shall have been reset.
-}

{- PK-97
The starting type of a partition shall be WARM or COLD. (Req 8)

We track partition start events, from both kernel and partition perspective
-}
\begin{circus}
\circchannel pk97fail
\end{circus}


{- PK-100
The partitioning kernel shall allow an authorized partition to start any partition (STOPPED mode to ACTIVE mode)  with a WARM or COLD starting type. (Req 9)
-}

{- 4.3.... not yet in here... --------------------------- -}
