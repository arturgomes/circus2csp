% {- =============================================================================
%
% 	HWEvents.csp

\section{HWEvents}
 These are the globally visible hardware events

 For now we cover a range of abstraction levels.
 These may be factored out to seperate source files later.

 We start with the most detailed first,
 and then present slimmed-down variants

% -}


\subsection{Global Clock}% ------------------------------------------------------------ -}
\begin{circus}
\circchannel tick\\  % standard clock tick
\circchannel tock  % used if 2-phase cycle is required
\end{circus}

\subsection{Main (Memory) Bus} % -----------------------------------------------------------
  We do not model bus operations as atomic, but rather as 2-phase:
	\begin{itemize}
   \item{1} signal direction and address
   \item{2} data transfer
	 \end{itemize}

\subsubsection{Addresses}
\begin{circus}
A == \{0,1\}
\end{circus}

\subsubsection{Data}
\begin{circus}
D == \{0,1\}
\end{circus}

\subsubsection{Direction, from CPU perspective}
\begin{circus}
Dir ::= R  % - reading (data from memory to CPU)
        | W  % - writing (data from CPU to memory)
\end{circus}
\begin{circus}

\circchannel ma : Dir \cross A\\  % address on memory bus
\circchannel md : Dir \cross D  %  data on memory bus
\end{circus}


\subsection{Interrupt Controller} --------------------------------------------------------
The IRQ controller takes in interrupt events (iXXX)
and decides when to drop, queue or raise them (rXXX)


\subsubsection{Timer}
\begin{circus}
\circchannel iTMR, rTMR
\end{circus}

\subsubsection{Memory Management Unit}
\begin{circus}

\circchannel iMMU, rMMU
\end{circus}

\subsubsection{External Devices}
\begin{circus}

\circchannel iDEV, rDEV
\end{circus}

\subsubsection{Software Traps}
\begin{circus}
\circchannel iTRP, rTRP
\end{circus}

\subsection{Machine Instructions, and operands} %--------------------------------------- -}
\begin{circus}

I ::=\\
  NOP\\   % generic instruction
  | S0\\    % set S bit to zero (user mode)
  | S1\\    % set S bit to one (supervisor mode)

OP == \{0 \upto 1\}

\end{circus}

\subsection{Processor Modes: User, Supervisor} %---------------------------------------- -}
\begin{circus}
PM ::=\\
 USR\\   % processor has just entered user mode
 | SUP   % processor has just entered supervisor mode

\circchannel pm : PM   % observable mode changes
\end{circus}
