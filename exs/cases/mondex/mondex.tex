
\begin{circus}
INDEX == \nat %-- {* Card identifiers *}
\also MONEY == \{0 \upto 5\} %-- {* Monetary amounts. *}
\end{circus}
In the paper money is represented as a nat, here we use an int so that we
have the option of modelling negative balances. This also eases proof as
integers form an algebraic ring.

\begin{circus}
\circchannel pay : INDEX \cross INDEX \cross MONEY\\ % -- {* Request a payment between two cards *}
\circchannel transfer : INDEX \cross INDEX \cross MONEY\\ % -- {* Effect the transfer *}
\circchannel accept : INDEX\\ %| -- {* Accept the payment *}
\circchannel reject : INDEX %-- {* Reject it *}
\end{circus}
% type_synonym action_mdx = "(st_mdx, ch_mdx) action"
\begin{circus}

\circprocess Card \circdef  i : INDEX \circspot\\
\circbegin\\
\circstate SCard \defs[value: INDEX]\\
Init \circdef (value := 0)\\
Credit \circdef  n: INDEX \circspot(value := (value + n))\\
Debit \circdef n: INDEX \circspot \lcircguard n \leq value \rcircguard \circguard (value := (value - n))\\
Transfer \circdef\\
  pay.i?j?n \then\\
    ( \lcircguard n > value \rcircguard \circguard reject!i \then \Skip\\
    \extchoice \lcircguard n \leq value \rcircguard \circguard transfer.i.j!n \then accept!i \then Debit(n) )\\
Receive \circdef transfer?j.i?n \then Credit(n)\\
Cycle \circdef ( Transfer \extchoice Receive ) \circseq Cycle\\
\circspot Init \circseq Cycle\\
\circend
\end{circus}
