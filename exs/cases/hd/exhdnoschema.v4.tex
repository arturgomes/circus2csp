\section{Circus Specification - Safety Requirements}
\subsection{Types and state variables - To be continued}
%<*zedtypes>
\begin{circus}
NatValue == 0 \upto 2\\
% BOOL ::= TRUE~|~FALSE~|~prepPhase~|~endPhase\\
% BOOL ::= TRUE~|~FALSE~|~FALSE\\
% BOOL ::= TRUE~|~ACETATE~|~BICARBONATE\\
TIME~==~\{~h:\nat;~m:\nat~@~(h,m)~\}\\
% BOOL ::= TRUE~|~ALARM~|~FALSE\\

 BOOL ::= TRUE | FALSE\\


chkInterval == 1\\
\end{circus}%<*hdchannels>
\begin{circus}
  \circchannel preparationPhase, therapyInitiation\\
  \circchannel connectingToPatient, duringTherapy, therapyEnding, endTreatment\\
  \circchannel autSelfTest, atrialTubing\\
  \circchannel ventricularTubing, connectDialyzer, setBloodLines
\end{circus}

%--responses to requirements
\begin{circus}
  \circchannel stopBloodFlow, produceAlarmSound, stopBP\\
  \circchannel disconnectDF, stopFlowDialyzer, stopCoagulantFlow
\end{circus}

\begin{circus}
  \circchannel fillArterialDrip, connPatientArterially, connPatientVenously\\
  \circchannel insertHeparinSyringe, heparinLineIsVented\\
  \circchannel connectingConcentrate~:~BOOL\\
  \circchannel salineBagLevel~:~\nat
\end{circus}

\begin{circus}
  % rinsingTime~:~\nat\\
\circchannel tick\\
\circchannel getCurrentTime~:~\nat\\
  \circchannel senAirVolLimit~:~\nat\\
  \circchannel senAirVol~:~\nat\\
  \circchannel senApTransdPress~:~\nat\\
  \circchannel senbloodFlowInEBC~:~\nat\\
  \circchannel setbloodFlowInEBC~:~\nat\\
  \circchannel setBloodFlow~:~\nat\\
  \circchannel senBypassVol~:~BOOL\\
  \circchannel senFflowDirect~:~BOOL\\
  \circchannel senHDMode~:~BOOL\\
  \circchannel senInfVol~:~\nat\\
  \circchannel senLastNonZeroBF~:~\nat\\
  \circchannel senNetFluidRemovalRate~:~\nat\\
  \circchannel senNetFluidRemoval:BOOL\\
  \circchannel setMinUFRateTreat~:~BOOL\\
  \circchannel senRotDirectBP~:~BOOL~\\
  \circchannel senRotDirectUFP~:~BOOL~\\
  \circchannel senSADSensorFlow~:~\nat\\
  \circchannel senVolInEBC~:~\nat\\
  \circchannel senVpTransdPress~:~\nat\\
  \circchannel setRinsingBPSpeed~:~\nat\\
  \circchannel setRinsingBPRate~:~\nat\\


\end{circus}
\begin{circus}
  % \circchannel inputOfSetRinsingParameters~:~\nat \cross \nat \cross \nat \cross \nat \cross \nat \cross \nat\\
  \circchannel inputOfSetDFParameters~:~\nat \cross BOOL \cross \nat\cross \nat \cross \nat\\
  % \circchannel inputOfSetUFParameters~:~\nat \cross \nat\cross \nat\cross \nat\\
  % \circchannel inputOfSetPressureParameters~:~\nat \cross \nat \cross BOOL \cross \nat  \cross BOOL\\
  \circchannel inputOfSetHeparinParameters~:~TIME \cross \nat \cross \nat \cross BOOL \cross \nat\\
\end{circus}
%</zedtypes>
\begin{circus}
\circchannelset HDMachineChanSet ==
\lchanset senApTransdPress,
senbloodFlowInEBC,
senBypassVol,
senFflowDirect,
senInfVol,
senLastNonZeroBF,
senNetFluidRemoval,
senNetFluidRemovalRate,
senRotDirectBP,
senRotDirectUFP,
senVolInEBC,
senSADSensorFlow,
senVpTransdPress,
senHDMode,
setMinUFRateTreat\rchanset\\
\end{circus}

\begin{circus}
\circchannelset TherapyPhaseChanSet ==
\\\t2 \lchanset preparationPhase, therapyInitiation, connectingToPatient,
\\\t3  duringTherapy, therapyEnding \rchanset
\end{circus}

% \begin{schema}{SysClockS}
%  time : \nat
% \end{schema}
%
% \begin{circus}%
%   \circprocess SysClock~\circdef~\circbegin\\
%   \circstate SysClockSt \defs SysClockS
%   \\ResetClock~\circdef~\t2 (time := 0) \circseq Clock\\
%   \\Clock~\circdef~
%   \\\t2 \circmu~X \circspot
%     ((\circif time \leq 3 \circthen (tick \then (time := time+1) \interleave (getCurrentTime!time \then \Skip))
%         \circelse time > 2 \circthen ((time := 0) \circseq \Skip)
%         \circfi) \circseq X)
%   \circspot ResetClock
%   \circend
% \end{circus}
\begin{schema}{HDGenComp}
    airVolLimit~:~\nat\\
    airVol~:~\nat\\
    alarm~:~BOOL\\
    % artBolusVol~:~\nat\\
    apTransdPress~:~\nat\\
    bloodFlowInEBC~:~\nat\\
    bypassValve~:~BOOL\\
    fflowDirect~:~BOOL\\
    hdActivity~:~\power~BOOL\\
    hdMachineState~:~\power~BOOL\\
    hdMode~:~BOOL\\
    infSalineVol~:~\nat\\
    lastNonZeroBF~:~\nat\\
    lowerPressureLimit~:~\nat\\
    netFluidRemovalRate~:~\nat\\
    netFluidRemoval~:~BOOL\\
    rotDirectionBP~:~BOOL\\
    rotDirectionUFP~:~BOOL~\\
    safeUpperLimit~:~\nat\\
    timerIntervalR9~:~\nat\\
    timerIntervalR10~:~\nat\\
    timerIntervalR11~:~\nat\\
    timerIntervalR12~:~\nat\\
    timerIntervalR13~:~\nat\\
    time~:~\nat\\
    upperPressureLimit~:~\nat\\
    volumeInEBC~:~\nat\\
    vpTransdPress~:~\nat\\
    sadSensorFlow~:~\nat\\
    % bloodLines~:~BOOL\\
    % minUFRateTreat~:~BOOL\\
    signalLamp~:~BOOL
\end{schema}

% \begin{schema}{RinsingParameters}
% fillingBPRate~:~NatValue\\
% rinsingBPRate~:~NatValue\\
% rinsingTime~:~NatValue\\
% ufRateForRinsing~:~NatValue\\
% ufVolForRinsing~:~NatValue\\
% bloodFlowForConnectingPatient~:~NatValue
  % \where
    % fillingBPRate~\in~\{~x~:~\nat | 0~\leq~x~\leq~6000\}\\
    % rinsingBPRate~\in~\{~x~:~\nat | 2~\leq~x~\leq~300\}\\
    % rinsingTime~\in~\{~x~:~\nat | 0~\leq~x~\leq~59\}\\
    % ufRateForRinsing~\in~\{~x~:~\nat | 2~\leq~x~\leq~3000\}\\
    % ufVolForRinsing~\in~\{~x~:~\nat | 2~\leq~x~\leq~2940\}\\
    % bloodFlowForConnectingPatient~\in~\{~x~:~\nat | 2~\leq~x~\leq~600\}\\
% \end{schema}
\begin{schema}{DFParameters}
    % conductivity~:~\nat\\
    bicarbonateAcetate~:~BOOL\\% \comment{I don't see invariants for this one}\\
    % bicarbonateConductivity~:~\nat\\
    dfTemperature~:~\nat\\
    % rinsingTime~:~\nat\\
    dfFlow~:~\nat\\
  \where
    % conductivity~\in~\{~x~:~\nat | 1~\leq~x~\leq~160\}\\
    % bicarbonateConductivity~\in~\{~x~:~\nat | 2~\leq~x~\leq~4\}\\
    dfTemperature~\in~\{~x~:~\nat | 2~\leq~x~\leq~40\}\\
    % rinsingTime~\in~\{~x~:~\nat | 0~\leq~x~\leq~59\}\\
    dfFlow~\in~\{~x~:~\nat | 20~\leq~x~\leq~800\}~\cup~\{~x~:~\nat | 2~\leq~x~\leq~300\}\\
\end{schema}
% \begin{schema}{UFParameters}
    % ufVol~:~\nat\\
    % therapyTime~:~\nat\\
    % minUFRate~:~\nat\\
    % maxUFRate~:~\nat\\
  % \where
    % ufVol~\in~\{~x~:~\nat | 3~\leq~x~\leq~20000\}\\
    % therapyTime~\in~\{~x~:~\nat | 2~\leq~x~\leq~600\}\\
    % minUFRate~\in~\{~x~:~\nat | 0~\leq~x~\leq~500\}\\
    % maxUFRate~\in~\{~x~:~\nat | 0~\leq~x~\leq~20\}\\
% \end{schema}
% \begin{schema}{PressureParameters}
%     limitDeltaMinMaxAP~:~\nat\\
%     actualTMPMaxTMP~:~\nat\\
%     limitsTMP~:~BOOL\\
%     lowHigh~:~\nat\\
%     extendedTMPLimitRange~:~BOOL\\
% \where
%     limitDeltaMinMaxAP~\in~\{~x~:~\nat | 2~\leq~x~\leq~3\}\\
%     actualTMPMaxTMP~\in~\{~x~:~\nat | 20~\leq~x~\leq~20\}\\
%     lowHigh~\in~\{~x~:~\nat | 2~\leq~x~\leq~99\}\\
% \end{schema}

% \begin{schema}{HeparinParameters}
    % heparinStopTime~:~TIME\\
    % heparinBolusVol~:~\nat\\
    % heparinProfileRate~:~\nat\\
    % treatmentWithoutHeparin~:~BOOL\\
    % syringeType~:~\nat
  % \where
    % heparinStopTime~\in~\{\,h,m:\nat | (h,m) \in TIME \land m<60~\land~h\leq10\,\}\\
    % heparinBolusVol~\in~\{~x~:~\nat | 1~\leq~x~\leq~3\}\\
    % heparinProfileRate~\in~\{~x~:~\nat | 1~\leq~x~\leq~3\}\\
    % syringeType~\in~\{10,20,30\}
% \end{schema}%</zedstate>
\begin{circus}%
  \circprocess HDMachine~\circdef~ \circbegin
  \circstate HDState \defs HDGenComp
                          % \land~RinsingParameters
                          \land~DFParameters
                          % \land~UFParameters
                          % \land~PressureParameters
                          % \land~HeparinParameters
%</hdchannels>%<*hdchannels>
% \begin{circus}
% \circchannel bfSensor : Flow \cross Time
% \end{circus}

  \\HDGenCompInit~\circdef~
    (airVolLimit~:=~0)\circseq
      (airVol~:=~0)\circseq
      (alarm~:=~FALSE)\circseq
      (apTransdPress~:=~0)\circseq
      (bloodFlowInEBC~:=~0)\circseq
      (bypassValve~:=~FALSE)\circseq
      (fflowDirect~:=~TRUE)\circseq
      (hdActivity~:=~\{FALSE\})\circseq
      (hdMachineState~:=~\{FALSE\})\circseq
      (infSalineVol~:=~0)\circseq
      (lastNonZeroBF~:=~0)\circseq
      (time~:=~0)\circseq
      (netFluidRemovalRate~:=~0)\circseq
      (netFluidRemoval~:=~FALSE)\circseq
      (rotDirectionBP~:=~TRUE)\circseq
      (rotDirectionUFP~:=~TRUE)\circseq
      (safeUpperLimit~:=~0)\circseq
      (timerIntervalR9~:=~0)\circseq
      (timerIntervalR10~:=~0)\circseq
      (timerIntervalR11~:=~0)\circseq
      (timerIntervalR12~:=~0)\circseq
      (timerIntervalR13~:=~0)\circseq
      (lowerPressureLimit~:=~0)\circseq
      (upperPressureLimit~:=~0)\circseq
      (volumeInEBC~:=~0)\circseq
      (vpTransdPress~:=~0)\circseq
      (hdMode~:=~TRUE)\circseq
      (signalLamp~:=~TRUE)%\circseq
      % (bloodLines~:=~FALSE)%\circseq (minUFRateTreat:=FALSE)
  \\Wait~\circdef~ \circvar n : \nat \circspot
   \circif n> 0 \circthen (tick \then Wait (n-1))
   \circelse n~=~0 \circthen \Skip
   \circfi\\

   	StopBloodFlow \circdef\\ stopBloodFlow \then \Skip

   	StopBP \circdef\\ stopBP \then \Skip

   	RaiseAlarm \circdef\\
   		(alarm~:=~TRUE) \circseq
   		(produceAlarmSound \then \Skip)

   	StopFlowDialyzer \circdef\\stopFlowDialyzer \then \Skip

   	DisconnectDF \circdef\\disconnectDF \then \Skip

   	StopCoagulantFlow \circdef\\ stopCoagulantFlow \then \Skip

   	%  Requirements
   	% R1

   	R1 \circdef\\
   		% getHdActivity?hdActivity \then\\
   		% getInfSalineVol?infSalineVol \then\\
   		(\circif (hdActivity = \{TRUE\}
   				\land infSalineVol > 0)
   			\circthen (StopBloodFlow \interleave RaiseAlarm)
   			\circelse (\lnot (hdActivity = \{TRUE\}
     				\land infSalineVol > 0))
            \circthen (tick \then R1) \circfi)
     			% \circthen (Wait(CheckInterval) \circseq R1) \circfi)

   	% subsubsection{Blood pump.}
   	% R2

   	NoFlowWatchDog \circdef\\
   		% getCurrentTime?time \then\\
   		% getLastNonZeroBF?lastNonZeroBF \then\\
   			((\lcircguard time - lastNonZeroBF > 1 \rcircguard \circguard tick \then StopBP \circseq RaiseAlarm)
   			\extchoice (\lcircguard time - lastNonZeroBF \leq 1 \rcircguard \circguard tick \then NoFlowWatchDog))

   	BloodFlowSample  \circdef\\
   	   % senbloodFlowInEBC?bloodFlowInEBC \then\\
   	   % getCurrentTime?time \then\\
   	  ((\circif bloodFlowInEBC \neq 0
   	   \circthen  tick \then (lastNonZeroBF:=time)
   	   \circelse \lnot bloodFlowInEBC \neq 0
    	   \circthen tick \then (bloodFlowInEBC:=0) \circfi ) \circseq BloodFlowSample)

   	R2 \circdef NoFlowWatchDog \interleave BloodFlowSample

   	% R3

   	R3 \circdef\\
   		% getHdMachineState?hdMachineState \then\\
   		% getbloodFlowInEBC?bloodFlowInEBC \then getDfFlow?dfFlow \then\\
   				\circif (hdMachineState = \{FALSE\} \land bloodFlowInEBC < ((dfFlow * 1) \mod 1))
   				\circthen RaiseAlarm
   				\circelse \lnot (hdMachineState = \{FALSE\} \land bloodFlowInEBC < ((dfFlow * 1) \mod 1))
   				\circthen (tick \then R3) \circfi

   	% R4

   	R4 \circdef\\
   		% getHdMachineState?hdMachineState \then\\
   		% getRotDirectUFP?rotDirectionBP \then\\
   		(\circif hdMachineState = \{FALSE\}
   				\land rotDirectionBP = FALSE
   			\circthen (StopBP \interleave RaiseAlarm)
   			\circelse (\lnot (hdMachineState = \{FALSE\}
        \land rotDirectionBP = FALSE)) \circthen (tick \then R4) \circfi)
     				% \land rotDirectionBP = FALSE)) \circthen (Wait(CheckInterval) \circseq R4) \circfi)

   	% subsubsection{Blood-side entry pressure.}
   	% R5
    %
   	R5 \circdef\\
   		% getHdMachineState?hdMachineState \then\\
   		% getVpTransdPress?vpTransdPress \then\\
   		% getUpperPressureLimit?upperPressureLimit \then\\
   		(\circif ((hdMachineState = \{FALSE\}) \land vpTransdPress > upperPressureLimit)
   			\circthen (StopBP \interleave RaiseAlarm)
   			\circelse (\lnot (hdMachineState = \{FALSE\} \land vpTransdPress > upperPressureLimit)) \circthen (tick \then  R5) \circfi)
   	% % R6
    %
   	R6 \circdef\\
   		% getHdMachineState?hdMachineState \then\\
   		% getVpTransdPress?vpTransdPress \then\\
   		% getLowerPressureLimit?lowerPressureLimit \then\\
   		(\circif (hdMachineState = \{FALSE\} \land ((vpTransdPress < lowerPressureLimit) \lor (apTransdPress > upperPressureLimit)))
   			\circthen (StopBP \interleave RaiseAlarm)
   			\circelse (\lnot (hdMachineState = \{FALSE\} \land vpTransdPress < lowerPressureLimit)) \circthen (tick \then  R6)\circfi)
    %
   	% % R7
    %
   	R7 \circdef\\
   		% getHdMachineState?hdMachineState \then\\
   		% getApTransdPress?apTransdPress \then\\
   		% getUpperPressureLimit?upperPressureLimit \then\\
   		(\circif (hdMachineState = \{FALSE\} \land apTransdPress > upperPressureLimit)
   			\circthen (StopBP \interleave RaiseAlarm)
   			\circelse (\lnot (hdMachineState = \{FALSE\} \land apTransdPress > upperPressureLimit)) \circthen (tick \then R7)\circfi)

   	% % R8
    %
   	% R8 \circdef\\
   	% 	getHdMachineState?hdMachineState \then\\
   	% 	getApTransdPress?apTransdPress \then\\
   	% 	getLowerPressureLimit?lowerPressureLimit \then\\
   	% 	(\circif hdMachineState == {FALSE}
   	% 			\land apTransdPress < lowerPressureLimit
   	% 		\circthen (StopBP \interleave RaiseAlarm)
   	% 		\circelse Wait(CheckInterval) \circseq R8)
    %
   	% % R9
    %
   	% TrackTimerVPTransdPressR9 \circdef\\
   	% 	((tick \then\\
   	% 		senVpTransdPress?x \then\\
   	% 		getTimerIntervalR9?timerIntervalR9 \then\\
   	% 		(\circif x > 0 \land timerIntervalR9 < 1
   	% 			\circthen setTimerIntervalR9!(timerIntervalR9+1) \then \Skip
   	% 			\circelse setTimerIntervalR9!0 \then \Skip  )) \circseq\\
   	% 			TrackTimerVPTransdPressR9)
    %
   	% R9internal \circdef\\
   	% 	getHdMachineState?hdMachineState \then\\
   	% 	senVpTransdPress?vpTransdPress \then\\
   	% 	getTimerIntervalR9?timerIntervalR9 \then\\
   	% 	(\circif hdMachineState == {TRUE}
   	% 			\land vpTransdPress > 0
   	% 			\land timerIntervalR9 > 0
   	% 		\circthen (StopBP \interleave RaiseAlarm)
   	% 		\circelse Wait(CheckInterval) \circseq R9internal)
    %
   	% R9 \circdef\\
   	% 	R9internal \interleave TrackTimerVPTransdPressR9
    %
   	% % R10
    %
   	% TrackTimerVPTransdPressR10 \circdef\\
   	%  (( tick \then\\
   	%  	senVpTransdPress?x \then\\
   	%  	getLowerPressureLimit?lowerPressureLimit \then\\
   	% 		getTimerIntervalR10?timerIntervalR10 \then\\
   	% 	(\circif x < lowerPressureLimit \land timerIntervalR10 < 1
   	% 		\circthen setTimerIntervalR10!(timerIntervalR10+1) \then \Skip
   	% 		\circelse setTimerIntervalR10!0 \then \Skip  ));
    %
   	% TrackTimerVPTransdPressR10)
    %
   	% R10internal \circdef\\
   	% 	getHdMachineState?hdMachineState \then\\
   	% 	senVpTransdPress?vpTransdPress \then\\
   	% 	getLowerPressureLimit?lowerPressureLimit \then\\
   	% 	getTimerIntervalR10?timerIntervalR10 \then\\
   	% 	(\circif hdMachineState == {TRUE}
   	% 		\land vpTransdPress < lowerPressureLimit
   	% 		\land timerIntervalR10 > 0
   	% 		\circthen (StopBP \interleave RaiseAlarm)
   	% 		\circelse Wait(CheckInterval) \circseq R10internal)
    %
   	% R10 \circdef\\
   	% 	R10internal \interleave TrackTimerVPTransdPressR10
    %
   	% % R11
    %
   	% TrackTimerAPTransdPressR11 \circdef\\
   	% 	(tick \then\\
   	%  		senApTransdPress?x \then\\
   	%  		getLowerPressureLimit?lowerPressureLimit \then\\
   	% 		getTimerIntervalR11?timerIntervalR11 \then\\
   	% 		 (\circif x < lowerPressureLimit \land timerIntervalR11 < 1
   	% 		 	\circthen setTimerIntervalR11!(timerIntervalR11+1) \then \Skip
   	% 		 	\circelse setTimerIntervalR11!0 \then \Skip  ));
    % 				TrackTimerAPTransdPressR11
    %
   	% R11internal \circdef\\
   	% 	getHdMachineState?hdMachineState \then\\
   	% 	senApTransdPress?apTransdPress \then\\
   	% 	getLowerPressureLimit?lowerPressureLimit \then\\
   	% 	getTimerIntervalR11?timerIntervalR11 \then\\
   	% 	(\circif hdMachineState == {TRUE}
   	% 			\land apTransdPress < lowerPressureLimit
   	% 			\land timerIntervalR11 > 0
   	% 		\circthen (StopBP \interleave RaiseAlarm)
   	% 		\circelse Wait(CheckInterval) \circseq R11internal)
    %
   	% R11 \circdef\\
   	% 	R11internal \interleave TrackTimerAPTransdPressR11
    %
   	% % R12
    %
   	% TrackTimerVPTransdPressR12 \circdef\\
   	% 	(tick \then\\
   	% 		senVpTransdPress?x \then\\
   	% 		getTimerIntervalR12?timerIntervalR12 \then\\
   	% 		(\circif x > 0 \land timerIntervalR12 < 1
   	% 			\circthen setTimerIntervalR12!(timerIntervalR12+1) \then \Skip
   	% 			\circelse setTimerIntervalR12!0 \then \Skip  )) \circseq\\
    % 				TrackTimerVPTransdPressR12
    %
   	% R12internal \circdef\\
   	% 	getHdActivity?hdActivity \then\\
   	% 	senVpTransdPress?vpTransdPress \then\\
   	% 	getTimerIntervalR12?timerIntervalR12 \then\\
   	% 	(\circif hdActivity == {reinfProcess}
   	% 			\land vpTransdPress > 0
   	% 			\land timerIntervalR12 > 0
   	% 		\circthen (StopBP \interleave RaiseAlarm)
   	% 		\circelse Wait(CheckInterval) \circseq R12internal)
    %
   	% R12 \circdef\\
   	% 	R12internal \interleave TrackTimerVPTransdPressR12
    %
   	% % R13
   	% TrackTimerVPTransdPressR13 \circdef\\
   	% 	(tick \then\\
   	% 		senApTransdPress?x \then\\
   	% 		getTimerIntervalR13?timerIntervalR13 \then\\
   	% 		(\circif x < 1 \land timerIntervalR13 < 1
   	% 			\circthen setTimerIntervalR13!(timerIntervalR13+1) \then \Skip
   	% 			\circelse setTimerIntervalR13!0 \then \Skip  )) \circseq\\
   	% 			TrackTimerVPTransdPressR13
    %
   	% R13internal \circdef\\
   	% 	getHdActivity?hdActivity \then\\
   	% 	senApTransdPress?apTransdPress \then\\
   	% 	getTimerIntervalR13?timerIntervalR13 \then\\
   	% 	(\circif hdActivity == {reinfProcess}
   	% 			\land apTransdPress > 0
   	% 			\land timerIntervalR13 > 0
   	% 		\circthen (StopBP \interleave RaiseAlarm)
   	% 		\circelse Wait(CheckInterval) \circseq R13internal)
    %
   	% R13 \circdef\\
   	% 	R13internal \interleave TrackTimerVPTransdPressR13
    %
   	% % subsubsection{Connecting the patient.}
   	% % R14
    %
   	% R14 \circdef\\
   	% 	getHdMachineState?hdMachineState \then\\
   	% 	getbloodFlowInEBC?bloodFlowInEBC \then\\
   	% 	getUpperPressureLimit?upperPressureLimit \then\\
   	% 	(\circif hdMachineState == {TRUE} \land bloodFlowInEBC == 0
   	% 		\circthen (StopBP \interleave RaiseAlarm)
   	% 		\circelse Wait(CheckInterval) \circseq R14)
    %
   	% % R15
    %
   	% R15 \circdef\\
   	% 	getHdMachineState?hdMachineState \then\\
   	% 	getVolInEBC?volumeInEBC \then\\
   	% 	getUpperPressureLimit?upperPressureLimit \then\\
   	% 	(\circif hdMachineState == {TRUE} \land volumeInEBC > 0
   	% 		\circthen (StopBP \interleave RaiseAlarm)
   	% 		\circelse Wait(CheckInterval) \circseq R15)
   	% % R16
    %
   	% R16 \circdef\\
   	% 	connectingToPatient \then Wait(1) \circseq therapyInitiation \then \Skip
    %
   	% % R17
    %
   	% R17 \circdef\\
   	% 	getHdMachineState?hdMachineState \then\\
   	% 	getRotDirectUFP?rotDirectionUFP \then\\
   	% 	getUpperPressureLimit?upperPressureLimit \then\\
   	% 	(\circif hdMachineState == {TRUE}
   	% 			\land rotDirectionUFP == FALSE
   	% 		\circthen (StopFlowDialyzer \interleave RaiseAlarm)
   	% 		\circelse Wait(CheckInterval) \circseq R17)
    %
   	% % subsubsection{Flow bicarbonate concentrate into mixing chamber.}
   	% % R18
    %
   	% R18 \circdef\\
   	% 	getHdActivity?hdActivity \then\\
   	% 	getBicarbonateAcetate?bicarbonateAcetate \then\\
   	% 	(\circif hdActivity == {preparationOfDF}
   	% 			\land bicarbonateAcetate == TRUE
   	% 		\circthen (DisconnectDF \interleave RaiseAlarm)
   	% 		\circelse Wait(CheckInterval) \circseq R18)
    %
   	% % R19
    %
   	% R19 \circdef\\
   	% 	getHdActivity?hdActivity \then\\
   	% 	getBicarbonateAcetate?bicarbonateAcetate \then\\
   	% 	(\circif hdActivity == {preparationOfDF}
   	% 			\land bicarbonateAcetate == ACETATE
   	% 		\circthen (DisconnectDF \interleave RaiseAlarm)
   	% 		\circelse Wait(CheckInterval) \circseq R19)
    %
   	% % subsubsection{Heat \land degas DF water.}
    %
   	% % R20
    %
   	% R20 \circdef\\
   	% 	getHdMachineState?hdMachineState \then\\
   	% 	getHdActivity?hdActivity \then\\
   	% 	getDfTemperature?dfTemperature \then\\
   	% 	(\circif hdMachineState == {prepPhase}
   	% 			\land hdActivity == {priming, FALSE}
   	% 			\land dfTemperature > 41
   	% 		\circthen (DisconnectDF \interleave RaiseAlarm)
   	% 		\circelse Wait(CheckInterval) \circseq R20)
    %
   	% % R21
    %
   	% R21 \circdef\\
   	% 	getHdMachineState?hdMachineState \then\\
   	% 	getDfTemperature?dfTemperature \then\\
   	% 	(\circif hdMachineState == {FALSE}
   	% 			\land dfTemperature < 33
   	% 		\circthen (DisconnectDF \interleave RaiseAlarm)
   	% 		\circelse Wait(CheckInterval) \circseq R21)
    %
   	% % subsubsection{Heparin.}
    %
   	% R22 \circdef\\
   	% 	getHdMachineState?hdMachineState \then\\
   	% 	getHdActivity?hdActivity \then\\
   	% 	getFflowDirect?fflowDirect \then\\
   	% 	(\circif hdMachineState == {prepPhase}
   	% 			\land hdActivity == {anticoagDelivRun}
   	% 			\land fflowDirect == FALSE
   	% 		\circthen (StopBloodFlow \interleave StopCoagulantFlow \interleave RaiseAlarm)
   	% 		\circelse Wait(CheckInterval) \circseq R22)
    %
   	% % subsubsection{Safety air detector.}
   	% % R23
    %
   	% MonitorSADSensorFlow \circdef\\
   	%  getSADSensorFlow?sadSensorFlow \then\\
   	%  	(\circif sadSensorFlow > 0
   	%  		\circthen (StopBP \interleave RaiseAlarm)
   	% 		\circelse R23)
    %
   	% R23 \circdef\\
   	% 	getHdMachineState?hdMachineState \then\\
   	% 	getHdActivity?hdActivity \then\\
   	% 	getFflowDirect?fflowDirect \then\\
   	% 	(\circif hdMachineState == {prepPhase, FALSE, TRUE}
   	% 			\land hdActivity == {TRUE, reinfProcess, anticoagDelivRun}
   	% 			\land fflowDirect == FALSE
   	% 		\circthen MonitorSADSensorFlow
   	% 		\circelse Wait(CheckInterval) \circseq R23)
   	% % R24
    %
   	% R24 \circdef\\
   	% 	senSADSensorFlow?sadSensorFlow \then\\
   	% 	(\circif sadSensorFlow >= 0 \land sadSensorFlow <= 2
   	% 		\circthen setAirVolLimit!1 \then \Skip
   	% 		\circelse Wait(CheckInterval) \circseq R24)
   	% % R25
    %
   	% R25 \circdef\\
   	% 	senSADSensorFlow?sadSensorFlow \then\\
   	% 	(\circif sadSensorFlow > 0 \land sadSensorFlow <= 1
   	% 		\circthen setAirVolLimit!1 \then \Skip
   	% 		\circelse Wait(CheckInterval) \circseq R25)
   	% % R26
    %
   	% R26 \circdef\\
   	% 	senSADSensorFlow?sadSensorFlow \then\\
   	% 	(\circif sadSensorFlow > 0
   	% 		\circthen setAirVolLimit!1 \then \Skip
   	% 		\circelse Wait(CheckInterval) \circseq R26)
   	% % R27
    %
   	% R27 \circdef\\
   	% 	tick \then\\
   	% 	senAirVol?airVol \then\\
   	% 	setAirVol!airVol \then R27
    %
   	% % R2832
    %
   	% R2832 \circdef\\
   	% 	getAirVol?airVol \then\\
   	% 	getAirVolLimit?airVolLimit \then\\
   	% 	getHdMachineState?hdMachineState \then\\
   	% 	getHdActivity?hdActivity \then\\
   	% 	getFflowDirect?fflowDirect \then\\
   	% 	(\circif airVol > airVolLimit
   	% 		\land (((hdActivity == {TRUE}
   	% 					\land hdMachineState == {prepPhase})
   	% 				or hdActivity == {TRUE,reinfProcess})
   	% 				or hdMachineState == {TRUE,FALSE})
   	% 		\circthen (StopBloodFlow \interleave RaiseAlarm)
   	% 		\circelse Wait(CheckInterval) \circseq R2832)
    %
   	% % subsubsection{Ultrafiltration.}
   	% % R33
    %
   	% R33 \circdef\\
   	% 	getNetFluidRemovalRate?netFluidRemovalRate \then\\
   	% 	getSafeUpperLimit?safeUpperLimit \then\\
   	% 	(\circif netFluidRemovalRate > safeUpperLimit
   	% 		\circthen (StopFlowDialyzer \interleave RaiseAlarm)
   	% 		\circelse Wait(CheckInterval) \circseq R33)
   	% % R34
    %
   	% R34 \circdef\\
   	% 	getHdMachineState?hdMachineState \then\\
   	% 	getNetFluidRemoval?netFluidRemoval \then\\
   	% 	getRotDirectUFP?rotDirectionUFP \then\\
   	% 	getHDMode?hdMode \then\\
   	% 	(\circif (hdMachineState == {FALSE} \land
   	% 		 netFluidRemoval == TRUE \land
   	% 		 rotDirectionUFP == FALSE \land
   	% 		 hdMode == FALSE)
   	% 		\circthen (StopFlowDialyzer \interleave RaiseAlarm)
   	% 		\circelse Wait(CheckInterval) \circseq R34)
   	% % R36
    %
   	% R36 \circdef\\
   	% 	getHdMachineState?hdMachineState \then\\
   	% 	getNetFluidRemoval?netFluidRemoval \then\\
   	% 	getBypassValve?bypassValve \then\\
   	% 	(\circif (hdMachineState == {FALSE} \land
   	% 		 netFluidRemoval == TRUE \land
   	% 		 bypassValve == TRUE)
   	% 		\circthen (StopFlowDialyzer \interleave RaiseAlarm)
   	% 		\circelse Wait(CheckInterval) \circseq R36)

   	ArterialBolusReq \circdef\\
   		R1

   	BloodPumpReq \circdef\\
   		R2% \interleave R3 \interleave R4

   	BloodEntryPressureReq \circdef\\
   		R5
   		\interleave R6 \interleave R7
   		% \interleave R8 \interleave R9
   		% \interleave R10 \interleave R11
   		% \interleave R12 \interleave R13

   	ConnPatientReq \circdef\\
   		R14
   		\interleave R15
   		\interleave R16
   		\interleave R17

   	FlowCarbMixChambReq \circdef\\
   		R18 \interleave R19

   	HeatDegasDFWaterReq \circdef\\
   		R20 \interleave R21

   	HeparinReq \circdef\\
   		R22

   	SafetyAirDetectorReq \circdef\\
   		R23
   		\interleave R24 \interleave R25
   		\interleave R26 \interleave R27
   		\interleave R2832

   	UltrafiltrationReq \circdef\\
   		R33 \interleave R34 \interleave R36


   %  {Therapy Processes}

   	%  {Therapy Preparation}

   	AutomatedSelfTest \circdef\\
   		autSelfTest \then (signalLamp:=TRUE)

   	ConnectingTheConcentrate \circdef\\
   		connectingConcentrate?x \then (bicarbonateAcetate:=x)\\

   	% SetRinsingParameters \circdef\\
   	% 	inputOfSetRinsingParameters?sFBPRate.sRBPRate.sRTime.sUFRFRinsing.sUFVFRinsing.sBFFCPatient \then\\
   	% 	(fillingBPRate:=sFBPRate) \circseq\\
   		% (rinsingBPRate:=sRBPRate) \circseq\\
   		% (rinsingTime:=sRTime) \circseq\\
   		% (ufRateForRinsing:=sUFRFRinsing) \circseq\\
   		% (ufVolForRinsing:=sUFVFRinsing)% \circseq\\
   		% (bloodFlowForConnectingPatient:=sBFFCPatient)

   	StdAtrialTubing \circdef\\
   		atrialTubing \then \Skip

   	StdVentricularTubing \circdef\\
   		ventricularTubing \then \Skip

   	InsertRinsingTubingSystem \circdef\\
   		(StdAtrialTubing \interleave StdVentricularTubing)

   	SalineBagLevels \circdef\\
   		salineBagLevel?ifs \then\\
   		(infSalineVol:=ifs)

   	% BloodLines \circdef\\
   	% 	(bloodLines:=FALSE)

   	% RinsingTesting \circdef\\ setRinsingBPRate?rbp \then
   	% 	(rinsingBPRate:=rbp)

   	InsertingRinsingTestingTubSystem \circdef\\
       InsertRinsingTubingSystem \circseq\\
       SalineBagLevels% \circseq\\ BloodLines% \circseq\\ RinsingTesting

   	PrepHeparinPump \circdef\\
   		insertHeparinSyringe \then\\
   		heparinLineIsVented \then \Skip

   	SetDFParameters \circdef\\
   		inputOfSetDFParameters?sCond.sBAc.sBCond.sDFTemp.sDFFlow \then\\
   	 	% (conductivity:=sCond) \circseq\\
   		(bicarbonateAcetate:=sBAc) \circseq\\
   		% (bicarbonateConductivity:=sBCond) \circseq\\
   		(dfTemperature:=sDFTemp) \circseq\\
   		(dfFlow:=sDFFlow)

   	% SetUFParameters \circdef\\
   		% inputOfSetUFParameters?sUFVol.sTTime.sMiUFRate.sMaUFRate \then\\
   		% (ufVol:=sUFVol) \circseq\\
   		% (therapyTime:=sTTime) \circseq\\
   		% (minUFRate:=sMiUFRate) \circseq\\
   		% (maxUFRate:=sMaUFRate)

   	% SetPressureParameters \circdef\\
   	% 	inputOfSetPressureParameters?sLDMMAP.sATMPMTMP.sLTMP.sLH.sETMPLR \then\\
   	% 	(limitDeltaMinMaxAP:=sLDMMAP) \circseq\\
   	% 	(actualTMPMaxTMP:=sATMPMTMP) \circseq\\
   	% 	(limitsTMP:=sLTMP) \circseq\\
   	% 	(lowHigh:=sLH) \circseq\\
   	% 	(extendedTMPLimitRange:=sETMPLR)

   	% SetHeparinParameters \circdef\\
   		% inputOfSetHeparinParameters?sHST.sHBV.sHPR.sTWH.sST \then\\
   		% (heparinStopTime:=sHST) \circseq\\
   		% (heparinBolusVol:=sHBV) \circseq\\
   		% (heparinProfileRate:=sHPR) \circseq\\
   		% (treatmentWithoutHeparin:=sTWH) \circseq\\
   		% (syringeType:=sST)

   	SetTreatmentParameters \circdef\\
        SetDFParameters%\circseq\\
        % SetUFParameters\circseq\\
        % SetPressureParameters %\circseq\\ SetHeparinParameters

   	RinsingDialyzer \circdef\\
   		connectDialyzer \then\\
   		fillArterialDrip \then\\
   		stopBP \then \Skip

   	TherapyPreparation \circdef\\
   	 	 preparationPhase \then\\
   	 	 AutomatedSelfTest \circseq\\
   	 	 ConnectingTheConcentrate \circseq\\
   	 	 % SetRinsingParameters \circseq\\
   	 	 InsertingRinsingTestingTubSystem \circseq\\
   	 	 PrepHeparinPump \circseq\\
   	 	 % SetTreatmentParameters \circseq\\
   	 	 RinsingDialyzer

   	%  {Therapy Initiation}

   	EnableUI \circdef\\
   		(signalLamp:=FALSE)

   	ConnectPatientArterially \circdef\\
   		connPatientArterially \then \Skip

   	ConnectPatientVenously \circdef\\
   		connPatientVenously \then \Skip

   	SetStopBloodFlow \circdef\\
   		setbloodFlowInEBC?bf \then (bloodFlowInEBC:=bf)

   	ConnectPatientStartTherapy \circdef\\
   	 		 connectingToPatient \then\\
   	 		 EnableUI \circseq\\
   	 		 ConnectPatientArterially \circseq\\
   	 		 SetStopBloodFlow \circseq\\
   	 		 ConnectPatientVenously \circseq\\
   	 		 (signalLamp:=TRUE) \circseq\\
   	 		 (hdMode:=TRUE)

   	MonitorBPLimits \circdef\\
   		\Skip

   	% TreatMinUFRate \circdef\\
   	% 	(minUFRateTreat:=TRUE)

   	HeparinBolus \circdef\\
   		\Skip

   	% ArterialBolus \circdef\\
   	% 	setArtBolusVol?artBolusVol \then \Skip

   	InterruptDialysis \circdef\\
   		(hdMode:=FALSE) \circseq\\
   		(signalLamp:=FALSE)

   	CompleteTreatment \circdef\\
   		therapyEnding \then\\
   		(signalLamp:=FALSE)

   	DuringTherapy \circdef\\
   			(MonitorBPLimits% \interleave TreatMinUFRate
   								 \interleave HeparinBolus
   								 % \interleave ArterialBolus
   								 \interleave InterruptDialysis) \circseq\\
    				CompleteTreatment

   	TherapyInitiation \circdef\\
   		therapyInitiation \then\\
   			ConnectPatientStartTherapy \circseq\\
   			DuringTherapy

   	%  {Therapy Ending}

   	Reinfusion \circdef\\
   		\Skip

   	EmptyingDialyzer \circdef\\
   		\Skip

   	EmptyingCartridge \circdef\\
   		\Skip

   	OverviewTherapy \circdef\\
   		\Skip

   	TherapyEnding \circdef\\
   			Reinfusion \circseq\\
   			EmptyingDialyzer \circseq\\
   			EmptyingCartridge \circseq\\
   			OverviewTherapy



      \\MainTherapy~\circdef
        TherapyPreparation \circseq
        TherapyInitiation \circseq
        TherapyEnding

      \\HDMachineMain~\circdef (%HDGenCompInit \circseq
      ((
      % MainTherapy
      % (MainTherapy \interleave  (ArterialBolusReq \interleave BloodEntryPressureReq))
      (MainTherapy \interleave
      (
      (R1 \interleave R2 \interleave R3 \interleave R4 \interleave R5 \interleave R6)
      % \interleave BloodPumpReq \interleave BloodEntryPressureReq
      ))
      % (MainTherapy \lpar \emptyset | MainTherapyChanSet | \emptyset \rpar  R1)
              % \lpar \emptyset | TherapyPhaseChanSet | \emptyset \rpar StatePhase)
              % \lpar \emptyset | HDMachineChanSet | \emptyset \rpar SensorReadings)
              )))
        \circspot HDMachineMain

\circend
\end{circus}
\begin{assert}
% "HDEnv = ( ( HDMachine(b_PAC1,b_PST1,b_HDM1,b_CON1,b_TIM1,b_BUT1,b_NAT1) [| {| tick,getCurrentTime |} |] SysClock(b_NAT1) ) \ {| tick,getCurrentTime |} )"
% \also "assert HDEnv :[deadlock free]"
% \also "assert HDEnv :[deadlock free]"
"MyHDMACHINE = let X = (HDMachine(b_PBO1,b_BOO1,b_NAT1)); X within X"
\also "assert MyHDMACHINE :[deadlock free]"
\end{assert}
