\section{PK-97} %{-  =======================================================================

       The starting type of a partition shall be WARM or COLD.

  Rationale: A transition entering in the ACTIVE mode corresponds to a start or
  a re-start of the partition. This (re-)start can be COLD or WARM. Whatever the
  starting type of the partition, the PK starts the execution of the partition
  from its entry start address. The only difference is the status of the
  associated starting type. This is the concern of the partition to access to
  its starting type status and to adapt its initialization strategy accordingly.

  Note: None

%============================================================================= -}
\begin{assert}
"include Platform.csp"
\end{assert}
 We track partition start events, from both kernel and partition perspective

\begin{circus}
\circchannel pk97fail\\

\circprocess PK97 \circdef \circbegin\\

  PK97knl \circdef ((fi.KNL?i?op \then PK97knl)  % kernel still running
     \extchoice (fi.P1?i?op \then PK97part)  % context switch to partition 1
     \extchoice (st.P1.CLD \then PK97cold)
     \extchoice (st.P1.WRM \then PK97warm))\\

  PK97part \circdef ((fi.KNL?i?op \then PK97knl)  % partition 1 interrupted
     \extchoice (fi.P1?i?op \then PK97part))\\  % partition 1 still running

  PK97cold \circdef (lbl.CLD \then fi?P1?i?op \then PK97warm)\\ % see a cold start,

  PK97warm \circdef (lbl.WRM \then fi?P1?i?op \then PK97part)\\ % see a warm start

  \circspot PK97knl\\ % assume the kernel is running
  \circend
\end{circus}
\begin{assert}
"assert PK97 :[divergence free]"
\also "assert PK97 :[deadlock free[F]]"
\also "PK97TEST(CODE) = (PK97 [| {| fi, st, lbl |} |] (PMM [| {| fi, hc |} |] CODE))"
\end{assert}
\subsection{Testing Good Behaviour} %------------------------------------------------ -}
\begin{circus}
\circprocess PK97GOODCODE \circdef \circbegin\\
PK97GOODCODEAct \circdef (fi.KNL.S0.0
  \then st.P1.CLD \then lbl.CLD \then fi.P1.NOP.0 \then lbl.WRM \then fi.P1.NOP.1
  \then fi.KNL.S0.0
  \then st.P1.WRM \then lbl.WRM \then fi.P1.NOP.1
  \then PK97GOODCODEAct)\\
 \circspot PK97GOODCODEAct\\
   \circend
\end{circus}
\begin{assert}
"assert PK97GOODCODE :[divergence free]"
\also "assert PK97GOODCODE :[deadlock free[F]]"
\also "PK97OK = PK97TEST(PK97GOODCODE)"
\also "assert PK97OK :[divergence free]"
\end{assert}
\subsubsection{TEST 1}
\begin{assert}
"assert PK97OK :[deadlock free[F]]  -- MUST SUCCEED !!"
\end{assert}

\subsection{Testing Bad Behaviour} %------------------------------------------------- -}
\begin{circus}
\circprocess PK97BAD1CODE \circdef \circbegin
  Act \circdef fi.KNL.S0.0 \then st.P1.CLD \then fi.P1.NOP.0 \then Act\\
  \circspot Act\\
 \circend\\
\circprocess PK97BAD2CODE \circdef\circbegin
  Act \circdef fi.KNL.S0.1 \then lbl.WRM \then fi.P1.NOP.1 \then Act\\
  \circspot Act\\
 \circend
 \end{circus}
\begin{assert}
"assert PK97BAD1CODE :[divergence free]"
\also "assert PK97BAD1CODE :[deadlock free[F]]"
\also "assert PK97BAD2CODE :[divergence free]"
\also "assert PK97BAD2CODE :[deadlock free[F]]"
\also "PK97BAD1 = PK97TEST(PK97BAD1CODE)"
\also "PK97BAD2 = PK97TEST(PK97BAD2CODE)"
\also "assert PK97BAD1 :[divergence free]"
\also "assert PK97BAD2 :[divergence free]"
\end{assert}
\subsubsection{TEST 2}
\begin{assert}
"assert PK97BAD1 :[deadlock free[F]]  -- MUST FAIL !!"
\also "assert PK97BAD2 :[deadlock free[F]]  -- MUST FAIL !!"
\end{assert}
