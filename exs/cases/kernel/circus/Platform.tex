% =============================================================================
\section{Platform}

 Here we model the behaviour of the hardware platform, initially with different
 platform models tailored for specific groups of requirements.
%
\begin{assert}
% "include HWEvents.csp" % TSPEvents already includes HWEvents
% \also
"include TSPEvents.csp"
\end{assert}


Things we need to observe:
  what code (kernel, partition1, ...) is currently running
  what priviledge mode (user, supervisor) the processor is currently in
  which code segment (initialisation, main loop, ...) of an entity is running,
    and when it starts and finishes.
  we need to observe when the kernel starts a partition (cold or warm)
  we also need to track partition modes and partition start hypercalls

starts in supervisor mode, executes code/hypercalls,
marking hardware state/mode changes as requested.
\begin{circus}
\circprocess PMM \circdef \circbegin Act \circdef pm.SUP \then Act \circspot Act \circend\\

\circprocess PMMRUN \circdef
\circbegin
  Act \circdef (fi?ent.S1?op \then pm.SUP \then Act
   \extchoice
   fi?ent.S0?op \then pm.USR \then Act
   \extchoice
   fi?ent.NOP?op \then Act
   \extchoice
   hc?p?call:HCALL \then pm.SUP \then Act)
   \circspot Act
\circend
\end{circus}
-- S1, S0 should result in an exception trap if attempted in user mode !!!!!

\begin{assert}
"assert PMM :[divergence free]"
\also "assert PMM :[deadlock free[F]]"
\end{assert}
