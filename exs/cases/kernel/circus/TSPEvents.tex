%  =============================================================================
%
% 	TSPEvents.tex
%
%  These are the globally visible TSP events
%
\section{TSPEvents}

\begin{assert}
"include HWEvents.csp"
\end{assert}

\subsection{Code Identity}% ---------------------------------------------------------------

Given configuration data we can identify the code running
by its range of instruction fetch addresses.
We abstract this as a type that identifies the relevant entity,
and by making code fetch observable.

For now we bake partition authorisation into our entity definition.

\begin{circus}
ENT ::=\\
  KNL\\  % kernel
 | P1\\   % partition 1 (authorised to do everything priviledged)
 | P2\\   % partition 2 (not authorised to anything priviledged)
PARTITION == \{P1, P2\}
\end{circus}

\subsection{Code Execution} %--------------------------------------------------------------

We model an instruction fetch with an event that records not just the opcode
and operands, but also the identity of the entity in whose memory space the
instruction resides.

\begin{circus}

\circchannel fi : ENT\cross I\cross OP
\end{circus}

\subsection{Partition Starting} %----------------------------------------------------------

In any partition code we simply mark the cold-start points (the beginning)
and the warm-start points (sometime later).
We also observe the kernel (or authorised partition) starting a partition.

Note that a switch to/from kernel and a partition can happen without these
markers - that is a normal context-switch.

\begin{circus}

LBL  % labels marking interesting code points
  ::= CLD   % in partition code just before first cold-start instruction
  | WRM   % in partition code just before first warm-start instruction
\end{circus}

\begin{circus}
\circchannel lbl : LBL\\
\circchannel st : ENT \cross LBL  % start entity, at label
\end{circus}
-- valid partition code has at most one of each.

\subsection{Partition Modes} % -------------------------------------------------------------
  These are STOPPED, READY, RUNNING, and ACTIVE = READY | RUNNING

\begin{circus}
PMODE ::= STOPPED | READY | RUNNING\\
ACTIVE == \{READY,RUNNING\} % is that right??
\end{circus}

\subsection{Hypercalls } % -------------------------------------------------------------
   We shall model these similarly to instruction fetches

\begin{circus}
HCALL ::= START \ldata PARTITION \rdata % start partition
\end{circus}
\begin{circus}
\circchannel hc : PARTITION \cross HCALL  % always followed by a kernel instruction
\end{circus}

 We need to define a check process that ensures that a hypercall is always
followed by a kernel instruction
\begin{circus}
	\circchannel badHCall
\end{circus}
\begin{circus}
\circprocess HCALLSCAN \circdef \circbegin\\
  Act \circdef fi?ent?i?op     \then  Act\\
              \extchoice  hc?p?call:HCALL \then  SEENHCALL\\
  SEENHCALL \circdef (fi.KNL?i?op     \then  Act\\
	             \extchoice  hc?p?call:HCALL \then  Act\\
	             \extchoice  (\Extchoice  p:PARTITION \circspot fi.p?i?op \then  HCALLFAIL))\\
  HCALLFAIL \circdef badHCall \then  \Stop\\
\circspot Act\\
\circend

% HCALLCHECK(CODE) = HCALLSCAN [| {|fi,hc|} |] CODE
% !!! Is this really part of some Requirement?
\end{circus}

\subsection{Kernel Actions}% --------------------------------------------------------------
  The Kernel can change partition modes,...

\begin{circus}
	\circchannel partmode : PMODE
\end{circus}
\subsection{Timings}%  --------------------------------------------------------------
  We can mark schedule boundaries

\begin{circus}
	SLOTBNDRY ::= SLOTBEGIN | SLOTEND\\
	\circchannel slot : PARTITION \cross SLOTBNDRY
\end{circus}

We should check that we always have slot begin followed by the end of the
same slot.
\begin{circus}

\circchannel badSchedule
\end{circus}
we assume we start outside any time-slot
\begin{circus}
\circprocess SCHEDSCAN \circdef \circbegin\\
  Act \circdef slot?p!SLOTBEGIN \then  SLOTSCAN(p)
              \extchoice  slot?p!SLOTEND   \then  SCHEDFAIL\\
  SLOTSCAN \circdef \circvar p : PARTITION \circspot
        (slot.p.SLOTEND   \then  Act
        \extchoice  (\Extchoice  p: PARTITION \cap \{p\} \circspot slot.p.SLOTEND \then  SCHEDFAIL)
        \extchoice  slot.p.SLOTBEGIN \then  SCHEDFAIL)\\
  SCHEDFAIL \circdef badSchedule \then \Stop\\
\circspot Act\\
\circend
\end{circus}
We should check that partition code only runs during its own timeslot
\begin{circus}

\circchannel partOutOfTime
\end{circus}
here we assume that SCHEDSCAN above is satisfied
\begin{circus}
\circprocess PTIMESCAN \circdef \circbegin\\
  Act \circdef fi?KNL?i?op  \then  Act
            \extchoice  (\Extchoice  p:PARTITION \circspot slot.p.SLOTBEGIN \then  PSLOT(p) )
            \extchoice  (\Extchoice  p:PARTITION \circspot fi.p?i?op        \then  PTIMEFAIL)\\
PSLOT  \circdef( \circvar p : PARTITION \circspot (fi?KNL?i?op    \then  PSLOT(p)
            \extchoice  fi.p?i?op      \then  PSLOT(p)
            \extchoice  slot.p.SLOTEND \then  Act
            \extchoice  (\Extchoice  p: PARTITION \cap \{p\} \circspot fi.p?i?op \then  PTIMEFAIL)))\\
PTIMEFAIL \circdef partOutOfTime \then  \Stop\\
\circspot Act\\
\circend
\end{circus}
this is definitely one of the other requirements!!!

\subsection{Code Termination} %------------------------------------------------------------
Partition code may end,
in which case we want to signal that and then busy loop (continue in the next subsection)
\begin{circus}
\circchannel done : ENT
\end{circus}


\subsection{Generic Partition Behaviour} %-------------------------------------------------

A partition starts with cold start code, followed by warm start code,
and then enters a running phase, from which it might decide to terminate.

\begin{circus}
\circprocess GP \circdef ent : ENT \circspot \circbegin\\
  Act \circdef lbl.CLD \then  fi.ent.NOP.0 \then  lbl.WRM \then  fi.ent.NOP.1 \then  GPRUN\\
GPRUN\circdef\\
  (fi.ent.NOP?op \then  GPRUN)\\
   \intchoice DONE\\
   DONE \circdef done.ent \then  NOPLOOP\\
   NOPLOOP \circdef fi.ent.NOP.0 \then  NOPLOOP   \\
   \circspot Act\\
\circend

\end{circus}
