\chapter{PK230 Demo (DASIA 2018)}

We simplify matters a lot.


Processor mode:
\begin{circus}
   PROCMODES ::= SUP | USR
\end{circus}

State variables, for now just the processor mode.
\begin{circus}
PROCMODE \defs [ procm : PROCMODES ]
\end{circus}

First, a way to identify the kernel and partitions.
For now we use numbers, with the kernel being -1,
0 denoting one authorised partition,
and 1 standing for an ordinary partition.
\begin{circus}
   KNL    == 0
\\ PRT    == 1
\\ CODEID == KNL \upto PRT
\end{circus}
\begin{assert}
"next(codeid) = (codeid+1) % 2"
\end{assert}

We shall use a global variable containing a $CODEID$ to indicate
what code is enabled.
\begin{circus}
RUNNING \defs [running:CODEID]
\end{circus}

Assume ``operations'' to switch processor mode ($UP$,$DOWN$),
do context switches ($SW$)
and and some other generic instruction ($GOP$).
and use a channel called $do$ to observe them.
\begin{circus}
   OPS ::= UP | DOWN | SW | GOP
\\ \circchannel do : CODEID \cross OPS
\\ \circchannel scream
\end{circus}


\begin{circus}
\circprocess\ Demo \circdef \circbegin \\
\t1 \circstate PMMState \defs PROCMODE \land RUNNING \\
\\
\t1 PMMInit \circdef (procm,running := SUP,KNL) \\
\\
\t1 PMMStep \circdef \\
\t2             do?i!UP \then (procm := SUP) \circseq \Skip \\
\t2 \extchoice  do?i!DOWN \then (procm := USR) \circseq \Skip \\
\t2 \extchoice  do?i!GOP \then \Skip \\
\t2 \extchoice  do?i!SW \then (running := next(running)) \circseq \Skip \\
\\
\t1 PK230Step \circdef \\
\t2 \lcircguard running = KNL \rcircguard \circguard (do?i?op \then \Skip)\\
\t2 \extchoice \lcircguard procm = USR \rcircguard \circguard (do?i?op \then \Skip)\\
\t2 \extchoice \lcircguard running = PRT \land procm = SUP \rcircguard \circguard (scream \then \Stop)\\
\\
\t1 RUN \circdef (PMMStep \interleave PK230Step) \circseq RUN \\
\\
\t1 \circspot PMMInit \circseq RUN \\
\circend
\end{circus}
