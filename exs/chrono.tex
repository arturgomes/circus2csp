\section{From \Circus\ to \CSPM\ through an example}
We describe the develpment of a modified version of a simple example from 
Oliveira's PhD thesis~\cite{MV05}: a chronometer. In fact, we use two models 
of a chronometer, a abstract model, $AChrono$ and a more concrete one, 
$Chrono$. For model checking purposes, we restrict the type $RANGE$ for values 
from 0 up to 2, instead of the conventional range form 0 up to 59, used for 
minutes and seconds. Three channels are used: the clock $tick$ every second; 
when the $time$ is requested, the channel $out$ outputs the minutes and 
seconds as a pair.
\begin{circus}
RANGE == 0 \upto 2
\also \circchannel tick,time
\also \circchannel out : \{min,sec : RANGE @ (min,sec)\}
\end{circus}
We start the construction of the \Circus\ process $AChrono$ with the 
definition of the state $AState$ with two state components: $sec$ and $min$, 
for seconds and minutes respectively.
\begin{circus}
\circprocess\ AChrono \circdef\\
\circbegin\\ 
\circstate AState \defs [ sec, min : RANGE ]\\
AInit \circdef (sec,min:=0,0)\\
IncSec \circdef (sec,min:=(sec+1)\mod 3,min)\\
IncMin \circdef (min,sec:=(min+1)\mod 3,sec)\\
Run \circdef \\

   (tick \then IncSec \circseq \\
    (\circif  sec = 0 \circthen IncMin
        \circelse sec \neq 0 \circthen \Skip \circfi ))\ 
    \extchoice (time \then out!(min, sec) \then \Skip) \\
\circspot (AInit \circseq (\circmu X \circspot (Run \circseq X)))\\
\circend
\end{circus}
Before translating the abstract representation of the chronometer $AChrono$ 
into \CSPM, we introduce a second model, more concrete of a chronometer. This 
time, we will separate the minutes and the seconds, each one in a separate 
action $Minutes$ and the $Seconds$. We therefore introduce three channels: 
$inc$ used by $Seconds$ to tell $Minutes$ to increment its minutes; at 
anytime, $Seconds$ can request the current minutes, through $minsReq$ and it 
will then output the current count for $mins$ and $sec$, through the channel 
$out$. We also define a \Circus\ channel set $Sync$ with the three above 
presented channels.
\begin{circus}
\circchannel inc, minsReq
\also \circchannel ans : RANGE
\also \circchannelset Sync == \lchanset inc, minsReq, ans \rchanset
\also \circprocess Chrono \circdef\ \circbegin\\
\circstate State \defs [ sec : RANGE] \land [min : RANGE ]\\
SecInit \circdef (sec:=0)\\
IncSec \circdef (sec:=(sec+1)\mod 3)\\
RunSec \circdef 
(tick \then IncSec \circseq
    (\circif  sec = 0 \circthen inc \then \Skip
    \circelse sec \neq 0 \circthen \Skip \circfi)
    \extchoice time \then minsReq \then ans?mins \then out!(mins,sec) 
      \then \Skip )  \\
Seconds \circdef SecInit \circseq (\circmu X \circspot (RunSec \circseq X))\\
MinInit \circdef (min:=0)\\
IncMin \circdef (min:=(min+1)\mod 3)\\
RunMin \circdef (inc \then IncMin) \extchoice (minsReq \then ans!min \then \Skip)\\
Minutes \circdef MinInit \circseq (\circmu X \circspot (RunMin \circseq X))\\
\circspot ((Seconds \lpar {seq} | Sync | {min} \rpar Minutes ) \circhide Sync)\\
\circend
\end{circus}
