\section{Models by Marcel Oliveira}

We list a number of models derived from a simple example from
Oliveira's PhD thesis~\cite{MV05}: a chronometer.

\subsection{$AChrono$}

A specification:
\begin{circus}
RANGE == 0 \upto 2
\also \circchannel tick,time
\also \circchannel out : \{min,sec : RANGE @ (min,sec)\}
\also
\circprocess\ AChrono \circdef\\
\circbegin\\
\circstate AState \defs [ sec, min : RANGE ]\\
AInit \circdef (sec,min:=0,0)\\
IncSec \circdef (sec,min:=(sec+1)\mod 3,min)\\
IncMin \circdef (min,sec:=(min+1)\mod 3,sec)\\
Run \circdef \\

   (tick \then IncSec \circseq \\
    (\circif  sec = 0 \circthen IncMin
        \circelse sec \neq 0 \circthen \Skip \circfi ))
    \extchoice (time \then out!(min, sec) \then \Skip) \\
\circspot (AInit \circseq (\circmu X \circspot (Run \circseq X)))\\
\circend
\end{circus}

\subsection{$Chrono$}
\begin{circus}
\circchannel inc, minsReq
\also \circchannel ans : RANGE
\also \circchannelset Sync == \lchanset inc, minsReq, ans \rchanset
\also \circprocess Chrono \circdef\ \circbegin\\
\circstate State \defs [ sec : RANGE ; min : RANGE ]\\
SecInit \circdef (sec:=0)\\
IncSec \circdef (sec:=(sec+1)\mod 3)\\
RunSec \circdef \t1
(tick \then IncSec \circseq
    (\circif  sec = 0 \circthen inc \then \Skip
    \circelse sec \neq 0 \circthen \Skip \circfi)
  \\\t3 \extchoice\\
  \t3  time \then minsReq \then ans?mins \then out!(mins,sec)
      \then \Skip )  \\
Seconds \circdef SecInit \circseq (\circmu X \circspot (RunSec \circseq X))\\
MinInit \circdef (min:=0)\\
IncMin \circdef (min:=(min+1)\mod 3)\\
RunMin \circdef (inc \then IncMin) \extchoice (minsReq \then ans!min \then \Skip)\\
Minutes \circdef MinInit \circseq (\circmu X \circspot (RunMin \circseq X))\\
\circspot ((Seconds \lpar {seq} | Sync | {min} \rpar Minutes ) \circhide Sync)\\
\circend
\end{circus}

\section{Models by Artur Gomes}

\subsection{$WakeUp$}

\begin{circus}
ALARM ::= ON | OFF\\
\circchannel snooze,radioOn\\

\circprocess\ WakeUp \circdef\\
\circbegin\\
\circstate WState \defs [ sec, min : RANGE; buzz:ALARM ]\\
WInit \circdef (sec,min,buzz:=0,0,OFF)\\
WIncSec \circdef (sec,min:=(sec+1)\mod 3,min)\\
WIncMin \circdef (min,sec:=(min+1)\mod 3,sec)\\
WRun \circdef \\
((tick \then WIncSec \circseq \\
		( \lcircguard sec = 0 \rcircguard \circguard WIncMin\\
		\extchoice \lcircguard sec \neq 0 \rcircguard \circguard \Skip\\
		\extchoice \lcircguard min = 1 \rcircguard \circguard radioOn \then (buzz:=ON)
		\extchoice (time \then out!(min, sec) \then \Skip)\\
		\extchoice (snooze \then (buzz:=OFF)))) \circhide \lchanset tick \rchanset)
\\
\circspot (WInit \circseq (\circmu X \circspot (WRun \circseq X)))\\
\circend
\end{circus}


MyWakeUp = (let X = WakeUpN; X within X)
assert MyWakeUp :[ deadlock free [FD] ]
assert WakeUpN :[ livelock free ]
assert WakeUpN :[ deterministic [FD] ]

\begin{circus}

\circprocess\ LocWakeUp \circdef\\
\circbegin\\
\circstate WState \defs [ sec, min : RANGE; buzz:ALARM ]\\
WInit \circdef \circvar ms : RANGE \circspot ((sec:=ms) \circseq (sec,min,buzz:=0,0,OFF))\\
WIncSec \circdef (sec,min:=(sec+1)\mod 3,min)\\
WIncMin \circdef (min,sec:=(min+1)\mod 3,sec)\\
WRun \circdef \\
((tick \then WIncSec \circseq \\
		( \lcircguard sec = 0 \rcircguard \circguard WIncMin\\
		\extchoice \lcircguard sec \neq 0 \rcircguard \circguard \Skip\\
		\extchoice \lcircguard min = 1 \rcircguard \circguard radioOn \then (buzz:=ON)
		\extchoice (time \then out!(min, sec) \then \Skip)\\
		\extchoice (snooze \then (buzz:=OFF)))) \circhide \lchanset tick \rchanset)
\\
\circspot (WInit \circseq (\circmu X \circspot (WRun \circseq X)))\\
\circend
\end{circus}

MyWakeUp = (let X = WakeUpN; X within X)
assert MyWakeUp :[ deadlock free [FD] ]
assert WakeUpN :[ livelock free ]
assert WakeUpN :[ deterministic [FD] ]
