\begin{zed}
   FREETYPEAXDEF ::= ELEM1AXDEF | ELEM2AXDEF
\end{zed}

\begin{zed}
   SETTYPEAXDEF ::= V0AXDEF | V1AXDEF | V2AXDEF
\end{zed}

\begin{axdef}
  axvaraxdef: \nat
\where
  axvaraxdef = 1
\end{axdef}

\begin{axdef}
      funintaxdef: \nat \pfun \nat
    \where
      funintaxdef~0 = 1 \land
      funintaxdef~1 = 0
\end{axdef}

\begin{axdef}
      fun: FREETYPEAXDEF \pfun \nat
    \where
      fun~ELEM1AXDEF = 0 \land
      fun~ELEM2AXDEF = 1
\end{axdef}

\begin{circus}
 \circchannel syncchanaxdef, renchanaxdef \\
 \circchannel pcawpacaxdef \\
 \circchannel syncchanaxdeftwo, syncchanaxdefthree \\
 \circchannel simplesyncchanaxdefonly : \nat \\
 \circchannel intchanaxdef : \nat \\
 \circchannel funchanaxdef : \nat \\
 \circchannel axchanaxdef : \nat \\
 \circchannel intchanaxdeftwo, intchanaxdefthree : \nat \\

 \circchannel setchanaxdef : SETTYPEAXDEF \\
 \circchannel freechanaxdef : FREETYPEAXDEF \\
 \circchannel prodchanaxdef : \nat \cross FREETYPEAXDEF \\
 \circchannel prodchanaxdeftwo : \nat \cross FREETYPEAXDEF \\
 \circchannel prodchanaxdefthree : \nat \cross \nat \cross \nat \\
 \circchannel [X, Y] genchanaxdef : X \cross Y \\
\end{circus}

%%%%%%GROUP 1%%%%%%%%

\begin{circus}
  \circchannelset Syncchanset == \lchanset syncchanaxdef \rchanset \\
\end{circus}

\begin{circus}
 \circprocess Axdefaxdef \circdef \circbegin \\
    \circspot axchanaxdef.axvaraxdef \then \Skip \\
 \circend \\
\end{circus}

\begin{circus}
 \circprocess BPComplexComm \circdef \circbegin \\
    \circspot prodchanaxdef?x?y \then prodchanaxdef!0!ELEM1AXDEF \then \Skip \\
 \circend \\
\end{circus}

\begin{circus}
\circprocess Proclocalconstantaxdef \circdef \circbegin \\
\end{circus}
 \begin{axdef}
  axvaraxdefinner: \nat
\where
  axvaraxdefinner = 0
\end{axdef}
\begin{circusaction}
\circspot intchanaxdef.axvaraxdefinner \then \Skip \\
\end{circusaction}
\begin{circus}
 \circend
\end{circus}

\begin{circus}
 \circprocess Funprocaxdef \circdef \circbegin \\
    \circspot funchanaxdef!(funintaxdef~1) \then \Skip \\
 \circend \\
\end{circus}

\begin{circus}
 \circprocess BPSchBoxTryaxdef \circdef \circbegin \\
    \circstate St ~~==~~ [~x, y : \nat | x \neq y~] \\
    \circspot \Skip \\
 \circend \\
\end{circus}

\begin{circus}
\circprocess BPSpecialCallaxdef \circdef \circbegin \\
    \circstate St ~~==~~ [~ xis, ypsilon : \nat ~] \\
    A \circdef syncchanaxdef \then \Skip \\
    \circspot A \\
 \circend \\
\end{circus}

\begin{circus}
 \circprocess BPSimpleSyncChanaxdef \circdef \circbegin \\
    \circspot simplesyncchanaxdefonly?x \then \Skip \\
 \circend \\
\end{circus}
\begin{circus}
 \circprocess BPSimpleReadChanaxdef \circdef \circbegin \\
    \circspot intchanaxdef?x \then \Skip \\
 \circend \\
\end{circus}

\begin{circus}
 \circprocess BPSimpleWriteChanaxdef \circdef \circbegin \\
    \circspot intchanaxdef!0 \then \Skip \\
 \circend \\
\end{circus}

\begin{circus} 
 \circprocess BPSimpleDotChanaxdef \circdef \circbegin \\
    \circspot freechanaxdef.ELEM1AXDEF \then intchanaxdef!0 \then \Skip \\
 \circend \\
\end{circus}

%%%%%%GROUP 2%%%%%%%%

\begin{circus}
 %\circprocess ParamProc \circdef x : SETTYPEAXDEF \circspot \circbegin \\ %PROBLEMATICO :: %DIGITE :: V0AXDEF
 %   \circspot setchanaxdef!x \then \Skip \\
 %\circend \\

 %\circprocess ParamProc2 \circdef x : SETTYPEAXDEF; y : SETTYPEAXDEF \circspot \circbegin \\ %PROBLEMATICO :: %DIGITE :: V0AXDEF
 %   \circspot setchanaxdef!x \then \Skip \\
 %\circend \\

 \circprocess BPParallelismVarDeclCommands \circdef \circbegin \\
    \circspot (\circvar bv : \nat \circspot renchanaxdef \then \Skip) \lpar | \lchanset syncchanaxdef \rchanset | \rpar (\circvar bw : \nat \circspot renchanaxdef \then \Skip) \\
 \circend \\

 \circprocess BPMultiSyncChan \circdef \circbegin \\
    \circspot (syncchanaxdef \then \Skip) \lpar | 
    Syncchanset
    | \rpar ((syncchanaxdef \then \Skip) \lpar | \lchanset syncchanaxdef \rchanset | \rpar (syncchanaxdef \then \Skip)) \\
 \circend \\

 \circprocess BPComplexComm2 \circdef \circbegin \\
    \circspot prodchanaxdef!0?z \then \Skip \\
\circend \\
 
 \circprocess BPExtChoiceComplexComm \circdef \circbegin \\
    \circspot (prodchanaxdef?x?y \then \Skip) \extchoice (prodchanaxdeftwo?x?y \then \Skip) \\
\circend \\

 \circprocess BPCommInputs \circdef \circbegin \\
   \circspot (prodchanaxdefthree?x?y?t \then \Skip) \extchoice (prodchanaxdefthree?z?w.2 \then \Skip) \\
 \circend \\

%Exercita a categoria sintática uX @ Proc (isto porque a acao abaixo serah transformada numa acao recursiva na chamada)
 \circprocess MuProc \circdef \circbegin \\
    A \circdef intchanaxdef!0 \then intchanaxdef!(funintaxdef~1) \then A \\
    \circspot A \\
 \circend \\

 \circprocess ProcCallActionWithParametersAxdef \circdef \circbegin \\
    A \circdef x : \nat ; y : \nat \circspot \circvar cx : \nat \circspot intchanaxdef!x \then intchanaxdef!y \then A (x - 1, y) \\
    \circspot A (2, 2) \\
 \circend \\
 
 \circprocess ProcCallActionWithParametersAndCircguards \circdef \circbegin \\
    A \circdef x : \nat ; y : \nat \circspot \lcircguard (\lnot (true \lor false)) \land x > y  \rcircguard \circguard pcawpacaxdef \then \lcircguard x > y \land x > y \rcircguard \circguard A (x - 1, y) \\
    \circspot A (1, 0) \\
 \circend \\

%Exercita circguards em interleaving forçado
 \circprocess ProcCircguardsForcedInterleaving \circdef \circbegin \\
    \circspot 
      (\lcircguard true \rcircguard \circguard pcawpacaxdef \then \Skip)
      \interleave
      (\lcircguard true \rcircguard \circguard pcawpacaxdef \then \Skip)
      \\
 \circend \\

%Exercita param actions com call actions
% \circprocess MuProc \circdef \circbegin \\
%    A \circdef intchanaxdef!0 \then A \\
%    B \circdef A
%    \circspot B \\
% \circend \\
\end{circus}
