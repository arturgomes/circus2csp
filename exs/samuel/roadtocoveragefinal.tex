%Program ::= CircusPara ∗ %COVERED
%CircusPara ::= Para | channel CDecl | chanset N == CSExpr | ProcDecl %COVERED
%CDecl ::= SimpleCDecl | SimpleCDecl; CDecl %(last syntactic category missed)
%SimpleCDecl ::= N+ | N+ : Expr | [N+] N+ : Expr | Schema-Exp %(two last syntactic categories missed)
%CSExpr ::= {|N ∗|} | N | CSExpr ∪ CSExpr | CSExpr ∩ CSExpr | CSExpr \ CSExpr | Appl %(Appl missed)
%ProcDecl ::= process N = ProcDef | process N [N+] = ProcDef %Last missed
%ProcDef ::= ParamProc | Proc %First missed
%ParamProc ::= Decl @ Proc | Decl 0 Proc | μ N • ParamProc %A ultima categoria sintatica aparece na dissertaçao de Angela mas nao na tese de Marcel. Alem disso nao ha 
%Proc ::=
%	begin PPara∗ state Para PPara∗ • Action end
%	Comm -> Proc | Pred & Proc | Proc \ CSExpr | Proc[N+ := N+] %(duas primeiras não aparecem na tese de Marcel e no Parser de Circus)
%	Proc ; Proc | Proc [] Proc | Proc |~| Proc | Proc |[ CSExpr || CSExpr ]| Proc %A ultima nao eh suportada pelo tradutor...
%	Proc |[ CSExpr ]| Proc | Proc ||| Proc | N | N(Expr+) | (Decl • Proc)(Expr+) %A ultima n eh suportada nem pelo parser
%	(μ N • Decl • Proc)(Expr + ) | N Expr + | (Decl Proc) Expr+ %As duas ultimas categorias sintaticas tem um L abrindo e um L fechando...
%	(μ N • Decl Proc) Expr + | N[Expr + ] | N[Expr + ](Expr + ) | N[Expr + ] Expr + % 
%
\begin{zed}
   FREETYPEFINAL ::= ELEM1FINAL | ELEM2FINAL
\end{zed}

\begin{zed}
   SETTYPEFINAL ::= V0FINAL | V1FINAL | V2FINAL
\end{zed}

%\begin{zed}
%   SETTYPEFINALSET ::= \{0, 1, 2\}
%\end{zed}

\begin{axdef}
  axvarfinal : \nat
\where
  axvarfinal = 1
\end{axdef}

%\begin{axdef}
%  axvarfinalprod: \power \nat% \cross \nat
%\where
%  axvarfinalprod = (1, 1)
%\end{axdef}

\begin{axdef}
      fun : FREETYPEFINAL \pfun \nat
    \where
      fun~ELEM1FINAL = 0 \land
      fun~ELEM2FINAL = 1
\end{axdef}

\begin{axdef}
      funintfinal : \nat \pfun \nat
    \where
      funintfinal~0 = 1 \land
      funintfinal~1 = 0
\end{axdef}

\begin{circus}
 \circchannel syncchanfinal, renchanfinal \\
 \circchannel pcawpacfinal \\
 \circchannel syncchanfinaltwo, syncchanfinalthree \\
 \circchannel simplesyncchanfinalonly : \nat \\
 \circchannel intchanfinal : \nat \\
 \circchannel funchanfinal : \nat \\
 \circchannel axchanfinal : \nat \\
 \circchannel intchanfinaltwo, intchanfinalthree : \nat \\
% \circchannel emptysetchanfinal : \{ \} \\
 \circchannel setchanfinal : SETTYPEFINAL \\
 \circchannel setchanfinal2 : SETTYPEFINAL \cross SETTYPEFINAL \\
 \circchannel freechanfinal : FREETYPEFINAL \\
 \circchannel prodchanfinal : \nat \cross FREETYPEFINAL \\
 \circchannel prodchanfinaltwo : \nat \cross FREETYPEFINAL \\
 \circchannel prodchanfinalthree : \nat \cross \nat \cross \nat \\
 \circchannel [X, Y] genchanfinal : X \cross Y \\
\end{circus}

\begin{circus}
 \circchannelset Emptychanstwo == \lchanset \rchanset \\
 \circchannelset Syncchanset == \lchanset syncchanfinal \rchanset \\
 \circchannelset Unarychans == \lchanset intchanfinal, setchanfinal, freechanfinal \rchanset \\
% \circchannelset Emptychans == \lchanset syncchanfinal \rchanset \\
 \circchannelset Prodchans == \lchanset prodchanfinal \rchanset \\
 \circchannelset Allchans == \lchanset intchanfinal, setchanfinal, freechanfinal, syncchanfinal, prodchanfinal \rchanset \\%Unarychans \cup Emptychans \cup Prodchans \\
 \circchannelset Emptychanset == \lchanset \rchanset \\ %Unarychans \cap Emptychans \\
 \circchannelset Diffchans == \lchanset freechanfinal \rchanset \\%Allchans \setminus Unarychans \\
 %\circchannelset FALTOU A CATEGORIA SINTATICA APPL
\end{circus}

%Exercita a categoria sintática Decl @ Proc
\begin{circus}

% \circprocess BPStopAction \circdef \circbegin \\
%    \circspot \Stop \\
% \circend \\
 
% \circprocess BPPrefixingAction \circdef BPSimpleSyncChanfinal \\
 
 %\circprocess BPGuardedAction \circdef \circbegin \\
    %\circspot \lcircguard (3 > (1 + 1)) \rcircguard \circguard (syncchanfinal \then \Stop) %\\
% \circend \\

%FROM HERE BELOW EVERYTHING WAS COMMENTED...

% \circprocess BPSeqAction \circdef \circbegin \\
%    A \circdef setchanfinal.0 \then \Skip \\
%    B \circdef setchanfinal.2 \then \Skip \\
%    \circspot A \circseq B \\
% \circend \\
 
% \circprocess BPExtChoiceAction \circdef \circbegin \\
%    A \circdef setchanfinal.0 \then \Skip \\
%    B \circdef setchanfinal.2 \then \Skip \\
%    \circspot A \extchoice B \\
% \circend \\

% \circprocess BPIntChoiceAction \circdef \circbegin \\
%    A \circdef setchanfinal.0 \then \Skip \\
%    B \circdef setchanfinal.2 \then \Skip \\
%    \circspot A \intchoice B \\
% \circend \\

 \circprocess BPParallelismWithNameSetsAction \circdef \circbegin \\
    \circstate St1 ~~==~~ [~ xis1, yp1 : SETTYPEFINAL ~] \\
    A1 \circdef %xis1, yp1 := V0FINAL, V2FINAL \circseq %COMMENTED TO AVOID SCALABILITY PROBLEMS ON THE CONSTRUCTION OF REFINEMENT TREE
      setchanfinal.xis1 \then \Skip \\
    B1 \circdef %xis1, yp1 := V0FINAL, V2FINAL \circseq %COMMENTED TO AVOID SCALABILITY PROBLEMS ON THE CONSTRUCTION OF REFINEMENT TREE
      setchanfinal.yp1 \then \Skip \\
    \circspot xis1, yp1 := V1FINAL, V1FINAL \circseq (A1 \lpar \{ xis1 \} | Allchans | \{ yp1 \} \rpar B1) \\
 \circend \\

 \circprocess BPParallelismPartitionTest \circdef \circbegin \\
    \circstate StPartition ~~==~~ [~ xisPartition, ypPartition : SETTYPEFINAL ~] \\
    APartition \circdef xisPartition, ypPartition := V0FINAL, V2FINAL \\
      %setchanfinal.xisPartition \then \Skip \\
    BPartition \circdef xisPartition, ypPartition := V0FINAL, V2FINAL \\
      %setchanfinal.ypPartition \then \Skip \\
    \circspot %xisPartition, ypPartition := V1FINAL, V1FINAL \circseq 
	(APartition \linter \{ xisPartition \} | \{ ypPartition \} \rinter BPartition) \circseq (setchanfinal2.xisPartition.ypPartition \then \Skip) \\
 \circend \\

 \circprocess BPInterleavingWithNameSetsAction \circdef \circbegin \\
    \circstate St2 ~~==~~ [~ xis2, yp2 : SETTYPEFINAL ~] \\
    A2 \circdef %xis2, yp2 := V0FINAL, V2FINAL \circseq %COMMENTED TO AVOID SCALABILITY PROBLEMS ON THE CONSTRUCTION OF REFINEMENT TREE
      setchanfinal.xis2 \then \Skip \\
    B2 \circdef %xis2, yp2 := V0FINAL, V2FINAL \circseq %COMMENTED TO AVOID SCALABILITY PROBLEMS ON THE CONSTRUCTION OF REFINEMENT TREE
      setchanfinal.yp2 \then \Skip \\
    \circspot xis2, yp2 := V1FINAL, V1FINAL \circseq (A2 \linter \{ xis2 \} | \{ yp2 \} \rinter B2) \\
 \circend \\

 \circprocess BPParallelismAction \circdef \circbegin \\
    \circstate St3 ~~==~~ [~ xis3, yp3 : SETTYPEFINAL ~] \\
    A3 \circdef xis3, yp3 := V0FINAL, V2FINAL \circseq setchanfinal.xis3 \then \Skip \\
    B3 \circdef xis3, yp3 := V0FINAL, V2FINAL \circseq setchanfinal.yp3 \then \Skip \\
    \circspot xis3, yp3 := V1FINAL, V1FINAL \circseq (A3 \lpar \{\} | Allchans | \{\} \rpar B3) \\
 \circend \\

 \circprocess BPParallActionWithGuardedBranches \circdef \circbegin \\
    \circspot (\lcircguard (\lnot (5 > 5)) \land true \rcircguard \circguard \Skip) \lpar | \lchanset intchanfinal, syncchanfinal \rchanset | \rpar (\lcircguard false \lor (\lnot (3 > 5)) \rcircguard \circguard \Skip) \\
 \circend \\

% \circprocess BPExtChoiceActionWithGuardedBranches \circdef \circbegin \\
%    \circspot (\lcircguard (\lnot (5 > 5)) \land true \rcircguard \circguard \Skip) \extchoice (\lcircguard false \lor (\lnot (3 > 5)) \rcircguard \circguard (\lcircguard false \rcircguard \circguard \Skip)) \\
% \circend \\

 \circprocess BPStateInvariant \circdef \circbegin \\
    \circstate St ~~==~~ [~ xis, yp : SETTYPEFINAL | true ~] \\
    \circspot \Skip \\
 \circend \\

 \circprocess BPBindTry \circdef \circbegin \\
    \circstate St ~~==~~ [~ xis, yp : SETTYPEFINAL | true ~] \\
    %A ~~==~~ [~ \Delta St; x?: \nat | xis = V0FINAL \land yp = V1FINAL ~]
    \circspot \Skip \\
 \circend \\

 \circprocess BPIffImpliesPredBranches \circdef \circbegin \\
    \circspot (\lcircguard true \iff true \rcircguard \circguard \Skip) \lpar | \lchanset intchanfinal, syncchanfinal \rchanset | \rpar (\lcircguard false \iff false \lor (true \implies false) \rcircguard \circguard \Skip) \\
 \circend \\

 \circprocess BPExpressions \circdef \circbegin \\
    \circspot intchanfinal.(\negate (\negate 1)) \then 
      intchanfinal.(
      (1 + (1 - (1 * (1 \div 1)))) \mod 2
      ) 
      \then \Skip \\
 \circend \\

 \circprocess BPBooleanExpressions \circdef \circbegin \\
    \circspot \circif (5 > 5) \circthen intchanfinal.2 \then \Skip \circelse (5 < 5) \circthen intchanfinal.1 \then \Skip \circelse (5 \geq 5) \circthen intchanfinal.0 \then \Skip \circelse (5 \leq 5) \circthen intchanfinal.3 \then \Skip \circfi \\
 \circend \\

 \circprocess BPStackVarEnv \circdef \circbegin \\
    \circspot intchanfinal?x \then (intchanfinal?x \then \Skip) \\
 \circend \\

 %[~ State~' | money' = 0 \land quantity' = cookieQuantity ~] 
% \circprocess CallIndexedProcExpr \circdef IndexProc \lcircindex 1 + 1 \rcircindex \\
% \circprocess [X] ProcessWithGenerics1 \circdef InterleaveProc \\
% \circprocess [X] ProcessWithGenerics2 \circdef ParamProc (V1FINAL) \\
% \circprocess [X] ProcessWithGenerics3 \circdef ProcessWithGenerics2 [\nat] \\
 
% \circprocess BPExtChoiceVarDeclCommand \circdef \circbegin \\
%    \circspot \circvar x : \nat \circspot syncchanfinal \then \Skip \extchoice \circvar x : \nat \circspot syncchanfinal \then \Skip\\
% \circend \\
 
% \circprocess BPVarDeclCommand \circdef \circbegin \\
%    \circspot \circvar x : \nat \circspot syncchanfinal \then \Skip \\
% \circend \\

% \circprocess BPNoChannelProc \circdef \circbegin \\
%  \circspot \Skip \\
% \circend \\
% \circprocess BPSSETTYPEFINALSET \circdef \circbegin \\
%    \circstate St ~~==~~ [~x : SETTYPEFINALSET ~] \\
%    \circspot setchanfinal2.0 \then \Skip \\
% \circend \\
 
% \circprocess BPConstDecl \circdef \circbegin \\
%    \circspot \circvar x : \nat \circspot syncchanfinal \then \Skip \\
% \circend \\
 
% \circprocess BPConstDecl \circdef \circbegin \\
%    \circspot \circcon h == 5 \circspot syncchanfinal \then \Skip \\
% \circend \\
 
% \circprocess BPHideAction \circdef \circbegin \\
%    \circspot intchanfinal.0 \then \Skip \circhide Unarychans \\
% \circend \\

%  \circprocess HideProc \circdef BasicProc1 \circhide \lchanset syncchanfinal \rchanset \\%Emptychans \\
 

% \circprocess BPCallActionWithParameters \circdef \circbegin \\
%    A \circdef x : \nat; y : \nat \circspot intchanfinal.(x - y) \then \Skip \\
%    \circspot A (3, 2) \\
% \circend \\

% \circprocess BPMuAction \circdef \circbegin \\
%    A \circdef x : \nat ; y : \nat \circspot intchanfinal.(x - y) \then \Skip \\
%    \circspot \circmu X \circspot A (3, 2) \circseq X \\
% \circend \\

 %\circprocess BPSeqActionIte \circdef \circbegin \\
    %\circspot \Semi x : SETTYPEFINAL \circspot freechanfinal.x \then \Skip %\\
 %\circend \\

 %\circprocess BPExtChoiceActionIte \circdef \circbegin \\
    %\circspot \Extchoice x : SETTYPEFINAL \circspot freechanfinal.x \then %\Skip \\
 %\circend \\

%\circprocess BPIntChoiceActionIte \circdef \circbegin \\
    %\circspot \Intchoice x : FREETYPEFINAL \circspot freechanfinal.x \then %\Skip \\
 %\circend \\

%\circprocess BPParallelNameSetsActionIte \circdef \circbegin \\
    %\circspot \lpar Emptychans \rpar x : FREETYPEFINAL \circspot \lpar %\{ \} \rpar freechanfinal.x \then \Skip \\
 %\circend \\

%\circprocess BPInterleaveActionIte \circdef \circbegin \\
    %\circspot \Interleave x : FREETYPEFINAL \circspot freechanfinal.x \then %\Skip \\
% \circend \\

%Replicados
% \circprocess ReplSeqProcess \circdef \Semi x : SETTYPEFINAL \circspot ParamProc (x) \\
% \circprocess ReplExtChoiceProcess \circdef \Extchoice x : SETTYPEFINAL \circspot ParamProc (x) \\
% \circprocess ReplIntChoiceProcess \circdef \Intchoice x : SETTYPEFINAL \circspot ParamProc (x) \\
% \circprocess ReplParallelProcess \circdef \Parallel x : SETTYPEFINAL \lpar Emptychans \rpar \circspot ParamProc (x) \\
% \circprocess ReplInterleaveProcess \circdef \Interleave x : SETTYPEFINAL \circspot ParamProc (x) \\

\end{circus}

%\begin{circus}
% \circprocess BPGenericChanfinal \circdef \circbegin \\
%  \circspot genchanfinal[FREETYPEFINAL, SETTYPEFINAL].ELEM1FINAL.V0FINAL \then \Skip \\
% \circend \\
%\end{circus}
