\begin{zed}
   FREETYPEITE ::= ELEM1ITE | ELEM2ITE
\end{zed}

\begin{zed}
   SETTYPEITE ::= V0ITE | V1ITE | V2ITE
\end{zed}

\begin{circus}
 \circchannel setchanprocessesite : SETTYPEITE \\
 \circchannel setchanprocessesite2 : FREETYPEITE \\
\end{circus}
\begin{circus}
\circprocess ParamProcIte \circdef x : SETTYPEITE \circspot \circbegin \\
   \circspot setchanprocessesite!x \then \Skip \\
\circend \\

\circprocess ParamProcIte2 \circdef x : FREETYPEITE \circspot \circbegin \\ %IDENTICAL TO PARAMPROCITE, BUT THIS ONE IS CONSTRUCT UPON FREETYPEITE, WHICH HAS ONLY 2 ELEMENTS OF FREE TYPE
   \circspot setchanprocessesite2!x \then \Skip \\
\circend \\

\circprocess ReplSeqProcess \circdef \Semi x : SETTYPEITE \circspot ParamProcIte (x) \\
\circprocess ReplExtChoiceProcess \circdef \Extchoice x : SETTYPEITE \circspot ParamProcIte (x) \\
\circprocess ReplIntChoiceProcess \circdef \Intchoice x : SETTYPEITE \circspot ParamProcIte (x) \\
\circprocess ReplParallelProcess \circdef \lpar \lchanset \rchanset \rpar x : FREETYPEITE \circspot ParamProcIte2 (x) \\
\circprocess ReplInterleaveProcess \circdef \Interleave x : FREETYPEITE \circspot ParamProcIte2 (x) \\
\end{circus}
