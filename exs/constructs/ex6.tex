\documentclass{llncs}
\usepackage{amssymb}
\usepackage{amsmath}
\usepackage{graphicx}
\usepackage{gensymb}
\usepackage{epstopdf}
\usepackage{hyperref}
\usepackage{circus}

\begin{document}
\begin{zed}
[NAME,TIME]\\
REPORT ~~::=~~ ok | already\_known | not\_known\\
\end{zed}
\begin{circus}
\circprocess ~SysClock ~\circdef~
\circbegin\\
  \circstate SysClockSt \defs [t : TIME]\\
  ResetClock ~\circdef~\Skip\\
  \circspot ResetClock\\
  \circend
\end{circus}
\begin{circus}
\circprocess ~Timer ~\circdef~
\circbegin\\
  \circstate TimerSt == [t : TIME]\\
  ResetClock ~\circdef~\Skip\\
  A1 ~\circdef~a \then b!2!3 \then c!3?x \then \Skip\\
  A2 ~\circdef~a?x \then a \then \Skip\\
  A3 ~\circdef~a \then \Skip\\
  \circspot a \then A1 \intchoice A2 \circseq A3\\
  \circend
\end{circus}

\begin{circus}
\circprocess ~MyInsaneProcess2 ~\circdef~ \Semi insane: TIME \circspot \\
		SysClock \extchoice Timer \circseq BothProcesses \intchoice BothProcesses1\\
\end{circus}
\begin{circus}

\circprocess ~MyInsaneProcess3 ~\circdef~
		Timer \lpar \lchanset \rchanset \rpar BothProcesses\\
\end{circus}
\begin{circus}

\circprocess ~MyInsaneProcess4 ~\circdef~
    SysClock \extchoice Timer \lpar \lchanset \rchanset \rpar BothProcesses \intchoice BothProcesses1\\
\end{circus}

\begin{circus}
\circprocess ~GuardTest ~\circdef~
\circbegin\\
  \circstate TimerSt == [t : TIME]\\
  ResetClock ~\circdef~\Skip\\
  A1 ~\circdef~\lcircguard 5>4 \rcircguard \circguard a \then b!2!3 \then c!3?x \then \Skip\\
  A2 ~\circdef~a?x \then a \then \Skip\\
  A3 ~\circdef~\circmu X \circspot a \then \Skip \circseq X\\
  A4 ~\circdef~\circmu X \circspot \circmu Y \circspot a \then \Skip \extchoice b \then \Skip \circseq Y \circseq X\\
  \circspot a \then A1 \intchoice A2 \circseq A3\\
  \circend
\end{circus}

% Process (
%   CProcess "GuardTest" (
%     ProcDef (
%       ProcMain (ZSchemaDef (ZSPlain "TimerSt") (ZSchema [Choose ("t",[]) (ZVar ("TIME",[]))])) 
%       [
%         CParAction "ResetClock" (
%           CircusAction CSPSkip
%         ),
%         CParAction "A1" (
%           CircusAction (
%             CSPGuard (ZMember (ZTuple [ZInt 5,ZInt 4]) (ZVar (">",[]))) 
%             (CSPCommAction (ChanComm "a" []) 
%               (CSPCommAction (ChanComm "b" [ChanOutExp (ZInt 2),ChanOutExp (ZInt 3)]) 
%                 (CSPCommAction (ChanComm "c" [ChanOutExp (ZInt 3),ChanInp "x"])
%                   CSPSkip)
%               )
%             )
%           )
%         ),
%         CParAction "A2" (
%           CircusAction (
%             CSPCommAction (ChanComm "a" [ChanInp "x"]) 
%               (CSPCommAction (ChanComm "a" []) CSPSkip))),
%         CParAction "A3" (
%           CircusAction (
%             CSPCommAction (ChanComm "a" []) CSPSkip))
%       ] 
%       (CSPIntChoice (
%         CSPCommAction (ChanComm "a" []) (CActionName "A1")) 
%         (CSPSeq 
%           (CActionName "A2") 
%           (CActionName "A3"))))))

\end{document}
