\section{Circus Specification - Safety Requirements}
\subsection{Types and state variables - To be continued}
%<*zedtypes>
\begin{zed}
[HOUR,MINUTE]
\also BUTTON ::= ON~|~OFF
\also SWITCH ::= ENABLED~|~DISABLED
\also DIRECTION ::= FORWARD~|~BACKWARDS
\also STATEPHASE ::= connectThePatient~|~initPhase~|~prepPhase~|~endPhase
\also ACTIVITY ::= rinsingEBC~|~applicationArterialBolus~|~anticoagDelivRun\\
 |~reinfProcess~|~preparationOfDF~|~priming~|~rinsing~|~idle
\also LID ::= OPEN~|~CLOSED
\also CONCENTRATE ::= ACID~|~ACETATE~|~BICARBONATE
\also TIME ~==~\{~h:HOUR;~m:HOUR~@~(h,m)~\}
\also HDMODE ::= RUN~|~ALARM~|~BYPASS
\also LAMP ::= GREEN~|~YELLOW
\also BOOL ::= TRUE~|~FALSE
\end{zed}


%<*hdchannels>
\begin{circus}
  \circchannel preparationPhase, therapyInitiation\\
  \circchannel connectingToPatient, duringTherapy, therapyEnding\\
  \circchannel autSelfTest, atrialTubing\\
  \circchannel ventricularTubing, connectDialyzer
\end{circus}

%--responses to requirements
\begin{circus}
  \circchannel stopBloodFlow, produceAlarmSound, stopBP\\
  \circchannel disconnectDF, stopFlowDialyzer, stopCoagulantFlow
\end{circus}

\begin{circus}
  \circchannel fillArterialDrip, connPatientArterially, connPatientVenously\\
  \circchannel insertHeparinSyringe, heparinLineIsVented\\
  \circchannel connectingConcentrate~:~CONCENTRATE\\
  \circchannel salineBagLevel~:~NAT
\end{circus}

\begin{circus}
  \circchannel senAirVol~:~NAT\\
  \circchannel senAirVolLimit~:~NAT\\
  \circchannel senApTransdPress~:~NAT\\
  \circchannel senBloodFlowInEBC~:~NAT\\
  \circchannel senBypassVol~:~LID\\
  \circchannel senFflowDirect~:~DIRECTION\\
  \circchannel senInfVol~:~NAT\\
  \circchannel senLastNonZeroBF~:~NAT\\
  \circchannel senNetFluidRemoval~:~SWITCH\\
  \circchannel senNetFluidRemovalRate~:~NAT\\
  \circchannel senRotDirectBP~:~DIRECTION~\\
  \circchannel senRotDirectUFP~:~DIRECTION~\\
  \circchannel senVolInEBC~:~NAT\\
  \circchannel senVpTransdPress~:~NAT\\
  \circchannel senSADSensorFlow~:~NAT\\
  \circchannel senHDMode~:~HDMODE\\
  \circchannel setMinUFRateTreat~:~BUTTON
\end{circus}

%</zedtypes>
\begin{circus}\circchannelset TherapyPhaseChanSet ==
\\ \lchanset preparationPhase, therapyInitiation, connectingToPatient,
\\ duringTherapy, therapyEnding \rchanset
\end{circus}
\begin{circus}
  \circchannelset SensorReadingsComm ==\\
  \\ \lchanset~senAirVol, senApTransdPress, \\
      senBloodFlowInEBC, senHDMode, \\
      senSADSensorFlow, senVpTransdPress~\rchanset
\end{circus}
\begin{circus}
  \circchannelset HDComm ==\\
  \\ \lchanset~getApTransdPress, getBloodFlowInEBC, \\
    getCurrentTime, getDfFlow, getHdActivity, \\
    getHdMachineState,getFflowDirect, \\
    getInfSalineVol, getVolInEBC,\\
    getLowerPressureLimit, getRotDirectUFP, \\
    getUpperPressureLimit, getDfTemperature, \\
    getVpTransdPress, senBloodFlowInEBC, \\
    senSADSensorFlow, setAirVol, \\
    setBloodFlowInEBC, setLastNonZeroBF, \\
    therapyInitiation, senVpTransdPress, \\
    getTimerIntervalR9, setTimerIntervalR9, \\
    getTimerIntervalR10,setTimerIntervalR10,\\
    getTimerIntervalR9~\rchanset
\end{circus}

\begin{circus}
  \circchannelset HDGenCompStChanSet ==\\
  \\ \lchanset~getAirVolLimit, getAirVol,
  getApTransdPress, getArtBolusVol,
  getBloodFlowInEBC, getBypassValve,
  getFflowDirect, getHdActivity,
  getHDMode, getInfSalineVol,
  getLastNonZeroBF, getLowerPressureLimit,
  getMinUFRateTreat, getNetFluidRemoval,
  getNetFluidRemovalRate, getSADSensorFlow,
  getSafeUpperLimit, getSignalLamp,
  getTimerIntervalR10, getTimerIntervalR11,
  getTimerIntervalR12, getTimerIntervalR13,
  getTimerIntervalR9, getUpperPressureLimit,
  getVolInEBC, getVpTransdPress,
  getBloodLines, setAirVolLimit,
  setAirVol, setAlarm,
  setBloodFlowInEBC, setLastNonZeroBF,
  setTimerIntervalR10, setTimerIntervalR11,
  setTimerIntervalR12, setTimerIntervalR13,
  setTimerIntervalR9, getRotDirectBP,
  getRotDirectUFP, preparationPhase,
  connectingToPatient, therapyInitiation,
  therapyEnding~\rchanset
\end{circus}
\begin{circus}
  \circchannelset RinsingParametersStChanSet ==\\
  \\ \lchanset~setFillingBPRate, setRinsingBPRate,
  setRinsingTime, setUfVolForRinsing,
  setBloodFlowForConnectingPatient,
  getRinsingBPRate, getRinsingTime,
  setUfRateForRinsing, getUfRateForRinsing,
  getUfVolForRinsing,
  getBloodFlowForConnectingPatient,
  getFillingBPRate~\rchanset
\end{circus}
\begin{circus}
  \circchannelset PressureParametersStChanSet ==\\
  \\ \lchanset~setLimitDeltaMinMaxAP, setActualTMPMaxTMP,
  setLimitsTMP, setLowHigh,
  setExtendedTMPLimitRange, getLimitDeltaMinMaxAP,
  getActualTMPMaxTMP, getLimitsTMP,
  getLowHigh, getExtendedTMPLimitRange ~\rchanset
\end{circus}
\begin{circus}
  \circchannelset HeparinParametersStChanSet ==\\
  \\ \lchanset~setTreatmentWithoutHeparin,
  setHeparinBolusVol, setHeparinProfileRate,
  getHeparinBolusVol, getHeparinProfileRate,
  getTreatmentWithoutHeparin, setSyringeType,
  getSyringeType~\rchanset
\end{circus}
\begin{circus}
  \circchannelset UFParametersStChanSet ==\\
  \\ \lchanset~setMaxUFRate, setMinUFRate,
  setTherapyTime, setUfVol,
  getMaxUFRate, getMinUFRate,
  getTherapyTime, getUfVol~\rchanset
\end{circus}
\begin{circus}
  \circchannelset DFParametersStChanSet ==\\
  \\ \lchanset~setBicarbonateAcetate,
  setBicarbonateConductivity, setDfFlow,
  setDfTemperature, setConductivity,
  getBicarbonateAcetate, getConductivity,
  getDfFlow, getDfTemperature,
  getBicarbonateConductivity~\rchanset
\end{circus}
\begin{circus}
  \circchannelset DFParametersStComm ==\\
  \\ \lchanset~setConductivity,setBicarbonateAcetate,\\
      setBicarbonateConductivity,\\
      setDfTemperature,setDfFlow~\rchanset
\end{circus}

\begin{circus}
  \circchannelset UFParametersStComm ==\\
  \\ \lchanset~setUfVol,setTherapyTime,\\
      setMinUFRate,setMaxUFRate~\rchanset
\end{circus}
\begin{circus}
  \circchannelset PressureParametersStComm ==\\
  \\ \lchanset~setLimitDeltaMinMaxAP,setActualTMPMaxTMP,\\
      setLimitsTMP,setLowHigh,setExtendedTMPLimitRange~\rchanset
\end{circus}
\begin{circus}
  \circchannelset HeparinParametersStComm ==\\
  \\ \lchanset~setHeparinStopTime,\\
      setHeparinBolusVol,\\
      setHeparinProfileRate,\\
      setTreatmentWithoutHeparin,\\
      setSyringeType~\rchanset
\end{circus}
\begin{circus}
  \circchannelset RinsingParametersStComm ==\\
  \\ \lchanset~setFillingBPRate,setRinsingBPRate,\\
    setRinsingTime,setUfRateForRinsing,\\
    setUfVolForRinsing,setBloodFlowForConnectingPatient\rchanset
\end{circus}

\begin{circus}
  \circchannelset MainTherapyChanSet ==
  \\ \lchanset~setActualTMPMaxTMP, setArtBolusVol,\\
    setBicarbonateAcetate, setBicarbonateConductivity,\\
    setBloodFlowForConnectingPatient, setBloodFlowInEBC,\\
    setBloodLines, setConductivity, setDfFlow,\\
    setDfTemperature,\\
    setExtendedTMPLimitRange, setFillingBPRate,\\
    setHDMode, setHeparinBolusVol,\\
     setHeparinProfileRate,\\
    setHeparinStopTime, setInfSalineVol,\\
    setLimitDeltaMinMaxAP, setLimitsTMP, setLowHigh,\\
    setMaxUFRate, setMinUFRate, setMinUFRateTreat,\\
    setRinsingBPRate,\\
    setRinsingTime, setSignalLamp,\\
    setSyringeType, setTherapyTime,\\
    setTreatmentWithoutHeparin,\\
    setUfRateForRinsing, setUfVol,\\
    setUfVolForRinsing,preparationPhase, therapyInitiation,\\
        connectingToPatient, duringTherapy,\\
        therapyEnding~\rchanset
\end{circus}


% \begin{circus}%
%   \circprocess ~SysClock ~\circdef~
%   \circbegin\\
%   \circstate SysClockSt \defs [time : NAT]\\
%   ResetClock \circdef (time:=0) \circseq Clock\\
%  Clock \circdef \circmu~X \circspot 
%  	tick \then (time := time+1) \interleave getCurrentTime!time \then \Skip \circseq X\\
%   Wait~\circdef~ \circvar n : NAT \circspot\\
%    \circif (n>0) \circthen tick \then Wait(n-1) \circelse (n=0) \circthen \Skip \circfi\\
%   \circspot ResetClock
%   \circend
% \end{circus}

\begin{circus}%
  \circprocess\ ~SysClock ~\circdef~\circbegin
\circstate SysClockSt \defs [time : NAT]\\
\end{circus}%
 \begin{circusaction}
  ResetClock ~\circdef~
  \\\t2time:=0 \circseq Clock\\
\end{circusaction}
\begin{circusaction}
  Clock ~\circdef~
  \\\t2 \circmu~X \circspot  (tick \then (time:=time+1) \interleave getCurrentTime!time \then \Skip)\circseq X
\end{circusaction}
\begin{circusaction}
  Wait~\circdef~ \circvar n : \num \circspot
   \circif n> 0 \circthen (tick \then Wait (n-1))
   \circelse n~=~0 \circthen \Skip
   \circfi\\
\end{circusaction}
\begin{circusaction}
  \circspot ResetClock
\end{circusaction}%
%\vspace{-1cm}
\begin{circus}
  \circend
\end{circus}

%</hdchannels>
\begin{schema}{HDGenComp}
    airVolLimit~:~NAT\\
    airVol~:~NAT\\
    alarm~:~SWITCH\\
    artBolusVol~:~NAT\\
    apTransdPress~:~NAT\\
    bloodFlowInEBC~:~NAT\\
    bypassValve~:~LID\\
    fflowDirect~:~DIRECTION\\
    hdActivity~:~\power~ACTIVITY\\
    hdMachineState~:~\power~STATEPHASE\\
    hdMode~:~HDMODE\\
    infSalineVol~:~NAT\\
    lastNonZeroBF~:~NAT\\
    lowerPressureLimit~:~NAT\\
    netFluidRemovalRate~:~NAT\\
    netFluidRemoval~:~SWITCH\\
    rotDirectionBP~:~DIRECTION\\
    rotDirectionUFP~:~DIRECTION~\\
    safeUpperLimit~:~NAT\\
    timerIntervalR9~:~NAT\\
    timerIntervalR10~:~NAT\\
    timerIntervalR11~:~NAT\\
    timerIntervalR12~:~NAT\\
    timerIntervalR13~:~NAT\\
    time~:~NAT\\
    upperPressureLimit~:~NAT\\
    volumeInEBC~:~NAT\\
    vpTransdPress~:~NAT\\
    sadSensorFlow~:~NAT\\
    bloodLines~:~BOOL\\
    minUFRateTreat~:~BUTTON
\end{schema}

\begin{schema}{RinsingParameters}
    fillingBPRate~:~NAT\\
    rinsingBPRate~:~NAT\\
    rinsingTime~:~NAT\\
    ufRateForRinsing~:~NAT\\
    ufVolForRinsing~:~NAT\\
    bloodFlowForConnectingPatient~:~NAT
  \where
    fillingBPRate~\in~\{~x~:~NAT | 0~\leq~x~\leq~6000\}\\
    rinsingBPRate~\in~\{~x~:~NAT | 50~\leq~x~\leq~300\}\\
    rinsingTime~\in~\{~x~:~NAT | 0~\leq~x~\leq~59\}\\
    ufRateForRinsing~\in~\{~x~:~NAT | 50~\leq~x~\leq~3000\}\\
    ufVolForRinsing~\in~\{~x~:~NAT | 50~\leq~x~\leq~2940\}\\
    bloodFlowForConnectingPatient~\in~\{~x~:~NAT | 50~\leq~x~\leq~600\}\\
\end{schema}
\begin{schema}{DFParameters}
    conductivity~:~NAT\\
    bicarbonateAcetate~:~CONCENTRATE\\% \comment{I don't see invariants for this one}\\
    bicarbonateConductivity~:~NAT\\
    dfTemperature~:~NAT\\
    rinsingTime~:~NAT\\
    dfFlow~:~NAT\\
  \where
    conductivity~\in~\{~x~:~NAT | 125~\leq~x~\leq~160\}\\
    bicarbonateConductivity~\in~\{~x~:~NAT | 2~\leq~x~\leq~4\}\\
    dfTemperature~\in~\{~x~:~NAT | 33~\leq~x~\leq~40\}\\
    rinsingTime~\in~\{~x~:~NAT | 0~\leq~x~\leq~59\}\\
    dfFlow~\in~\{~x~:~NAT | 300~\leq~x~\leq~800\}~\cup~\{~x~:~NAT | 50~\leq~x~\leq~300\}\\
\end{schema}
\begin{schema}{UFParameters}
    ufVol~:~NAT\\
    therapyTime~:~NAT\\
    minUFRate~:~NAT\\
    maxUFRate~:~NAT\\
  \where
    ufVol~\in~\{~x~:~NAT | 100~\leq~x~\leq~20000\}\\
    therapyTime~\in~\{~x~:~NAT | 10~\leq~x~\leq~600\}\\
    minUFRate~\in~\{~x~:~NAT | 0~\leq~x~\leq~500\}\\
    maxUFRate~\in~\{~x~:~NAT | 0~\leq~x~\leq~4000\}\\
\end{schema}
\begin{schema}{PressureParameters}
    limitDeltaMinMaxAP~:~NAT\\
    actualTMPMaxTMP~:~NAT\\
    limitsTMP~:~SWITCH\\
    lowHigh~:~NAT\\
    extendedTMPLimitRange~:~SWITCH\\
\where
    limitDeltaMinMaxAP~\in~\{~x~:~NAT | 10~\leq~x~\leq~100\}\\
    actualTMPMaxTMP~\in~\{~x~:~NAT | 300~\leq~x~\leq~700\}\\
    lowHigh~\in~\{~x~:~NAT | 2~\leq~x~\leq~99\}\\
\end{schema}

\begin{schema}{HeparinParameters}
    heparinStopTime~:~TIME\\
    heparinBolusVol~:~NAT\\
    heparinProfileRate~:~NAT\\
    treatmentWithoutHeparin~:~SWITCH\\
    syringeType~:~NAT
  \where
    heparinStopTime~\in~\{\,h,m:NAT | (h,m) \in TIME \land m<60~\land~h\leq10\,\}\\
    heparinBolusVol~\in~\{~x~:~NAT | 1~\leq~x~\leq~100\}\\
    heparinProfileRate~\in~\{~x~:~NAT | 1~\leq~x~\leq~100\}\\
    syringeType~\in~\{10,20,30\}
\end{schema}

\begin{circus}%
   \circchannelset SysClockCs == \lchanset tick, getCurrentTime \rchanset\\

\circprocess\ ~HDMachine ~\circdef~ \circbegin
\circstate HDState \defs HDGenComp 
						\land~RinsingParameters
						\land~DFParameters
						\land~UFParameters
						\land~PressureParameters
						\land~HeparinParameters
SensorReadings ~\circdef~
    %senAirVolLimit?airVolLimit \then SensorReadings\\
    senApTransdPress?apTransdPress \then SensorReadings\\
    \extchoice senBloodFlowInEBC?bloodFlowInEBC \then SensorReadings\\
    \extchoice senBypassVol?bypassValve \then SensorReadings\\
    \extchoice senFflowDirect?fflowDirect \then SensorReadings\\
    \extchoice senInfVol?infVol \then SensorReadings\\
    \extchoice senLastNonZeroBF?lastNonZeroBF \then SensorReadings\\
    \extchoice senNetFluidRemoval?netFluidRemoval \then SensorReadings\\
    \extchoice senNetFluidRemovalRate?netFluidRemovalRate \then SensorReadings\\
    \extchoice senRotDirectBP?rotDirectionBP \then SensorReadings\\
    \extchoice senRotDirectUFP?rotDirectionUFP \then SensorReadings\\
    \extchoice senVolInEBC?volumeInEBC \then SensorReadings\\
    \extchoice senSADSensorFlow?sadSensorFlow \then SensorReadings\\
    \extchoice senVpTransdPress?vpTransdPress \then SensorReadings\\
    \extchoice senHDMode?hdMode \then SensorReadings\\
    \extchoice setMinUFRateTreat?minUFRateTreat \then SensorReadings

  StatePhase \circdef\\
  (preparationPhase \then (hdMachineState:=prepPhase) \circseq StatePhase\\
          \extchoice connectingToPatient \then
          \\(hdMachineState:=connectThePatient) \circseq StatePhase\\
          \extchoice therapyInitiation
          \then (hdMachineState:=initPhase) \circseq StatePhase\\
          \extchoice therapyEnding \then (hdMachineState:=endPhase) \circseq StatePhase)

    InitState ~\circdef~ HDGenCompInit \circseq SensorReadings

StopBloodFlow~\circdef~ stopBloodFlow \then \Skip
%RaiseAlarm~\circdef~ ([~\Delta~HDState | alarm'~=~ENABLED ~]) \circseq produceAlarmSound \then \Skip\\
StopBP~\circdef~stopBP \then \Skip
StopFlowDialyzer~\circdef~ stopFlowDialyzer \then \Skip
DisconnectDF~\circdef~ disconnectDF \then \Skip
StopCoagulantFlow~\circdef~ stopCoagulantFlow \then \Skip


PreR1 \circdef
  [\Delta HDState | hdActivity~=~\{applicationArterialBolus\}\land infSalineVol~>~400]

  R1~\circdef~PreR1 \circseq (StopBloodFlow~\interleave~RaiseAlarm)
  \extchoice (\lnot PreR1) \circseq Wait(CheckInterval) \circseq R1


NoFlowWatchDog ~\circdef~ \\
    \lcircguard time - lastNonZeroBF > 120000 \rcircguard \circguard tick \then StopBP \circseq RaiseAlarm
    \\\extchoice~\lcircguard time - lastNonZeroBF \leq 120000\rcircguard \circguard tick \then NoFlowWatchDog

BloodFlowSample  ~\circdef~  \\ %second approach for R2
    senBloodFlowInEBC?bloodFlowInEBC \then
	\circif (bloodFlowInEBC \neq 0) \circthen ( lastNonZeroBF := time ) \circelse (bloodFlowInEBC~=~0) \circthen \Skip \circfi \circseq BloodFlowSample

R2 ~\circdef~ NoFlowWatchDog \interleave BloodFlowSample
%</R2>


%<*ZR3>
PreR3 \circdef [\Delta HDState | hdMachineState~=~\{initPhase\}\land bloodFlowInEBC < ((dfFlow * 70) \div 100) ]

  R3~\circdef~PreR3 \circseq RaiseAlarm \extchoice (\lnot PreR3) \circseq Wait(CheckInterval)\circseq R3
%</ZR3>

%<*ZR4>
PreR4 \circdef [\Delta HDState | hdMachineState~=~\{initPhase\} rotDirectionBP~=~BACKWARDS ] 

  R4~\circdef~PreR4~\circseq~ StopBP~\interleave~RaiseAlarm 
  \extchoice (\lnot PreR4) \circseq Wait(CheckInterval)\circseq R4
%<*ZR5>
PreR5 \circdef [\Delta  HDState | hdMachineState~=~\{initPhase\} \land vpTransdPress~>~upperPressureLimit ]


  R5~\circdef~PreR5~\circseq~StopBP~\interleave~RaiseAlarm
  \extchoice (\lnot PreR5) \circseq Wait(CheckInterval)\circseq R5

%</ZR5>

ArterialBolusReq~\circdef~R1
BloodPumpReq~\circdef~ R2~\interleave~R3~\interleave~R4
BloodEntryPressureReq~\circdef~(
                R5~\interleave~R6~\interleave~ R7~\interleave~
                \\R8~\interleave~ R9 \interleave~R10~\interleave~
                \\R11~\interleave~R12~\interleave~ R13)
ConnPatientReq~\circdef~R14~\interleave~R15~\interleave~R16~\interleave~R17
FlowCarbMixChambReq~\circdef~R18~\interleave~R19
HeatDegasDFWaterReq~\circdef~R20~\interleave~R21
HeparinReq~\circdef~R22
SafetyAirDetectorReq~\circdef~(
                R23~\interleave~R24~\interleave~ R25~\interleave~
                \\R26~\interleave~ R27 \interleave~R2832)
UltrafiltrationReq~\circdef~(
                R33~\interleave~R34~\interleave~
                \\R35~\interleave~ R36)

SoftwareRequirements~\circdef~
    \circmu~X \circspot
    ArterialBolusReq
              \\\interleave~BloodPumpReq
              \\\interleave~BloodEntryPressureReq
              \\\interleave~ConnPatientReq
              \\\interleave~FlowCarbMixChambReq
              \\\interleave~HeatDegasDFWaterReq
              \\\interleave~HeparinReq
              \\\interleave~SafetyAirDetectorReq
              \\\interleave~UltrafiltrationReq \circseq X\\

AutomatedSelfTest~\circdef~autSelfTest \then (signalLamp:=GREEN)

ConnectingTheConcentrate~\circdef~
 connectingConcentrate?bicarbonateAcetate \then \Skip


StdAtrialTubing~\circdef~atrialTubing~\then~\Skip
StdVentricularTubing~\circdef~ventricularTubing~\then~\Skip
InsertRinsingTubingSystem~\circdef~StdAtrialTubing \interleave StdVentricularTubing
SalineBagLevels~\circdef~salineBagLevel?infSalineVol~\then~\Skip
BloodLines~\circdef~setBloodLines \then (bloodLines:=TRUE)
RinsingTesting~\circdef~setRinsingBPSpeed?rinsingBPRate~\then~\Skip
InsertingRinsingTestingTubSystem~\circdef~
  \\ InsertRinsingTubingSystem\circseq
  \\ SalineBagLevels\circseq
  \\ BloodLines\circseq
  \\ RinsingTesting

PrepHeparinPump~\circdef~insertHeparinSyringe~\then~heparinLineIsVented~\then~\Skip


SetTreatmentParameters~\circdef~ SetDFParameters \circseq SetUFParameters \circseq
\\ SetPressureParameters \circseq SetHeparinParameters

RinsingDialyzer~\circdef~connectDialyzer~\then~fillArterialDrip~\then~stopBP~\then~\Skip

TherapyPreparation~\circdef
  \\ preparationPhase \then AutomatedSelfTest \circseq
  \\ ConnectingTheConcentrate \circseq SetRinsingParameters \circseq
  \\ InsertingRinsingTestingTubSystem \circseq PrepHeparinPump \circseq
  \\ SetTreatParameters \circseq RinsingDialyzer

EnableUI~\circdef~(signalLamp:=YELLOW)
ConnectPatientArterially~\circdef~connPatientArterially~\then~\Skip
ConnectPatientVenously~\circdef~connPatientVenously~\then~\Skip
SetStopBloodFlow~\circdef~setBloodFlow?bloodFlowInEBC~\then~\Skip
ConnectPatientStartTherapy~\circdef~
  \\ (hdMachineState:=connectThePatient) \circseq
  \\ EnableUI \circseq
  \\ ConnectPatientArterially \circseq
  \\ SetStopBloodFlow\circseq
  \\ ConnectPatientVenously \circseq
  \\ (signalLamp,hdMode:=GREEN,RUN)

MonitorBPLimits ~\circdef~\Skip
TreatMinUFRate ~\circdef~ setMinUFRateTreat?ON \then \Skip
HeparinBolus ~\circdef~\Skip
ArterialBolus~\circdef~setArtBolusVol?artBolusVol \then \Skip
InterruptDialysis~\circdef~ senHDMode!BYPASS \then (signalLamp:=YELLOW)
CompletTreatment~\circdef~ endTreatment \then (signalLamp:=YELLOW)
DuringTherapy~\circdef~
      \\ MonitorBPLimits\\
            \interleave TreatMinUFRate\\
            \interleave HeparinBolus\\
            \interleave ArterialBolus\\
            \interleave InterruptDialysis \circseq
      \\ CompletTreatment

TherapyInitiation~\circdef
\\therapyInitiation \then ConnectPatientStartTherapy
  \circseq DuringTherapy

Reinfusion~\circdef~\Skip
EmptyingDialyzer~\circdef~ \Skip
EmptyingCartridge~\circdef~\Skip
OverviewTherapy~\circdef~\Skip

TherapyEnding~\circdef
  therapyEnding~\then~
  Reinfusion
  \circseq EmptyingDialyzer
  \circseq \\ EmptyingCartridge
  \circseq OverviewTherapy
MainTherapy~\circdef
  TherapyPreparation
  \circseq TherapyInitiation
  \circseq TherapyEnding

HDMachine~\circdef HDGenCompInit \circseq
MainTherapy
        \lpar HDMachineChanSet \rpar  SoftwareRequirements
        \lpar TherapyPhaseChanSet\rpar StatePhase
        \lpar HDMachineChanSet \rpar SensorReadings\\
  \circspot HDMachine \lpar SysClockCs \rpar SysClock 
  \circend
\end{circus}
